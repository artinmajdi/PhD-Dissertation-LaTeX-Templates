\chapter{Automated Thalamic Nuclei Segmentation Using Multi-Planar Cascaded Convolutional Neural Networks}

\begin{abstract}
\textbf{Purpose:}
To develop a fast and accurate convolutional neural network based method for segmentation of thalamic nuclei.
\textbf{Methods}
A cascaded multi-planar scheme with a modified residual U-Net architecture was used to segment thalamic nuclei on conventional and white-matter-nulled (WMn) magnetization prepared rapid gradient echo (MPRAGE) data. A single network was optimized to work with images from healthy controls and patients with multiple sclerosis (MS) and essential tremor (ET), acquired at both 3T and 7T field strengths. WMn-MPRAGE images were manually delineated by a trained neuroradiologist using the Morel histological atlas as a guide to generate reference ground truth labels. Dice similarity coefficient and volume similarity index (VSI) were used to evaluate performance. Clinical utility was demonstrated by applying this method to study the effect of MS on thalamic nuclei atrophy.
\textbf{Results:}
Segmentation of each thalamus into twelve nuclei was achieved in under a minute. For 7T WMn-MPRAGE, the proposed method outperforms current state-of-the-art on patients with ET with statistically significant improvements in Dice for five nuclei (increase in the range of $0.05-0.18$) and VSI for four nuclei (increase in the range of $0.05-0.19$), while performing comparably for healthy and MS subjects. Dice and VSI achieved using 7T WMn-MPRAGE data are comparable to those using 3T WMn-MPRAGE data. For conventional MPRAGE, the proposed method shows a statistically significant Dice improvement in the range of $0.14-0.63$ over FreeSurfer for all nuclei and disease types. Effect of noise on network performance shows robustness to images with SNR as low as half the baseline SNR\@. Atrophy of four thalamic nuclei and whole thalamus was observed for MS patients compared to healthy control subjects, after controlling for the effect of parallel imaging, intracranial volume, gender, and age ($p < 0.004$).
\textbf{Conclusion:}
The proposed segmentation method is fast, accurate, performs well across disease types and field strengths, and shows great potential for improving our understanding of thalamic nuclei involvement in neurological diseases.
\textbf{KEYWORDS:\ } Convolutional neural network; Thalamic nuclei segmentation; Clinical analysis; White-matter-nulled MPRAGE
\end{abstract}



\section{INTRODUCTION}

The thalamus is a deep brain gray matter structure that relays information between various subcortical areas and the cerebral cortex~\cite{aggleton_Hippocampalanterior_2010} and plays a critical role in regulating sleep, consciousness, arousal, and awareness~\cite{steriade_Functional_1988,bodart_Coma_2013,stein_Functional_2000}. It is subdivided into multiple nuclei with varying functions. Thalamic involvement has been reported in schizophrenia~\cite{chen_Hippocampus_2017, parnaudeau_Mediodorsal_2018}, alcohol use disorder (AUD)~\cite{arts_Korsakoff_2017, fama_Thalamic_2014}, Parkinson's disease~\cite{benabid_Chronic_1993}, multiple sclerosis (MS)~\cite{minagar_Thalamus_2013}, essential tremor~\cite{fama_Thalamic_2015,benabid_Chronic_1996,brodkey_Tremor_2004} and Alzheimer's disease~\cite{braak_Alzheimer_1991}. These pathologies affect different thalamic nuclei differently and, therefore, accurate volumetry of thalamic nuclei can be beneficial for tracking disease progression and treatment efficacy~\cite{lee_Lateral_2016,braak_Alzheimer_1991}.

Manual delineation of thalamic nuclei from in-vivo scans is very tedious and requires specialized knowledge~\cite{jakab_Generation_2012,krauth_Mean_2010}.Due to low intra-thalamic contrast~\cite{tourdias_Visualization_2014}, thalamic nuclei are not easily distinguishable in conventional T1- and T2-weighted magnetic resonance images. As a result, most structural MRI based automated methods have only segmented the whole thalamus as part of subcortical brain segmentation~\cite{fischl_Whole_2002,freesurfer_FreeSurfer_2012,patenaude_Bayesian_2011,heckemann_Automatic_2006,iglesias_Probabilistic_2018}.

Diffusion tensor imaging (DTI) based methods that use either local or global properties of the diffusion tensor have been more popular for thalamic nuclei segmentation. Behrens~\cite{behrens_Noninvasive_2003} used tractography of cortical projections to the thalamus to segment thalamic regions, but this method requires precise knowledge of neuroanatomy to identify the relevant cortical regions. More automated, computationally efficient methods have been proposed that use k-means clustering of the dominant diffusion orientation to achieve thalamic parcellation~\cite{wiegell_Automatic_2003,kumar_Direct_2015,mang_Thalamus_2012}.To date, the most consistent DTI-based method~\cite{battistella_Robust_2017} uses spherical harmonic decomposition based orientation distribution functions to achieve robust segmentation of seven thalamic nuclei. However, the low spatial resolution of echo-planar imaging which underlies DTI and the predominance of gray matter in the thalamus which results in low anisotropy make these DTI-based methods suboptimal~\cite{su_Thalamus_2019}, often resulting in segmentation of only the larger thalamic groups. Advanced techniques such as susceptibility-weighted imaging (SWI)~\cite{haacke_Susceptibility_2004} can provide better intra-thalamic contrast and have been used for segmentation of thalamic nuclei at 7T~\cite{ abosch_Assessment_2010, xiao_Multimodal_2016}. However, these methods have found limited application in presurgical targeting, focusing mainly on the VIM nucleus. Hybrid approaches that combine DTI with T1 weighted imaging have also been proposed~\cite{glaister_Thalamus_2017,stough_Thalamic_2013}.

Recently, high spatial resolution structural MRI has been investigated for thalamic nuclei segmentation. The most widely used T1-weighted structural MRI sequence is magnetization prepared rapid gradient echo (MPRAGE), where the cerebrospinal fluid (CSF) is nulled. We refer to this method as CSFn-MPRAGE\@. Iglesias~\cite{iglesias_Probabilistic_2018} proposed a probabilistic atlas constructed using manual delineation of 26 thalamic nuclei per thalamus on six autopsy specimens and used Bayesian inference to segment 3T MPRAGE images into 26 nuclei per side~\cite{leemput_Encoding_2009,iglesias_Computational_2015}. However, this method is very time consuming, requiring multiple hours for the segmentation of one subject and has not been thoroughly validated against manual segmentation~\cite{iglesias_Probabilistic_2018}. Liu~\cite{ glaister_Thalamus_2017,liu_Thalamic_2015} segmented thalamic nuclei from 3T T1-weighted MRI data using an atlas developed from multiple MPRAGE and SWI sequences acquired at 7T. A multi-atlas label fusion and statistical shape modeling algorithm was used to transfer from 7T to 3T. Variants of the MPRAGE sequence have been proposed to better visualize the intra-thalamic structures~\cite{vassal_Direct_2012,sudhyadhom_High_2009}. Su~\cite{su_Thalamus_2019} used a WMn-MPRAGE sequence that is optimized for intra-thalamic contrast~\cite{tourdias_Visualization_2014} in conjunction with a multi-atlas technique, called thalamus optimized multi-atlas segmentation (THOMAS), to segment the thalamus into 12 nuclei. The performance of this based method hinges on the accuracy of a computationally expensive registration step~\cite{iglesias_Probabilistic_2018,alven_Improving_2017}.This method has only been validated on specialized WMn-MPRAGE data.

Convolutional neural networks (CNNs) are a class of deep learning techniques that use convolutional kernels to capture the non-linear mapping between an input image and its segmentation labels. Unlike atlas-based segmentation techniques, CNNs do not depend on image registration and manual feature extraction~\cite{zhu_Dilated_2019}. While many studies have explored the advantages of using CNNs for subcortical segmentation~\cite{brebisson_Deep_2015,moeskops_Automatic_2016,shakeri_Subcortical_2016,milletari_HoughCNN_2017}, those studies are limited to the whole thalamus. Due to  a paucity of training data and the high computational and memory requirements of 3D analysis, most proposed methods make use of 2D CNNs~\cite{shakeri_Subcortical_2016}. However, the use of 2D networks does not fully exploit the anatomical information present in 3D MRI data. Alternatively, multi-planar techniques that make use of 2D CNNs along the three orthogonal planes have been shown~\cite{moeskops_Automatic_2016,kushibar_Automated_2018,roth_New_2014,prasoon_Deep_2013} to improve segmentation performance with lower computational cost than a full 3D analysis. Transfer learning techniques have also been investigated to mitigate the lack of sufficient training data~\cite{tajbakhsh_Convolutional_2016}.

In this work, we propose the use of a modified residual U-Net in a cascaded multi-planar scheme for thalamic nuclei segmentation. We first demonstrate this method for WMn-MPRAGE~\cite{tourdias_Visualization_2014} data, evaluating it on data from 7T as well as 3T. We then extend this work to the more commonly acquired CSFn-MPRAGE by fine tuning the network trained on WMn-MPRAGE data. The performance of both networks is validated on healthy subjects and patients with MS and ET and compared to current state-of-the-art segmentation methods. Finally, robustness of the proposed method to SNR and its applicability to data from patients with multiple sclerosis are investigated.
