\chapter{Introduction}


\section{Introduction and Contextual Setting}
Machine learning is a powerful technological tool that has progressively delivered cutting-edge solutions to complex problems in a variety of fields. Its application spans a wide array, from data interpretation and natural language processing to contemporary breakthroughs in domains such as computer vision, radiology, neuroimaging, and biological image analysis. This dissertation delves into distinct use cases, demonstrating the adaptability and capabilities of machine learning in solving varied challenges. Machine learning has signified a notable advancement in computational capacity, thereby enabling us to tackle issues once deemed insurmountable or too intricate. In sectors such as healthcare, machine learning algorithms help in disease diagnosis, predicting patient outcomes, and personalizing treatment plans. The automotive industry, with the emergence of self-driving vehicles, is undergoing a transformation propelled by machine learning. Likewise, in crowdsourcing, machine learning aids in aggregating information gathered from a vast populace to solve complex problems or drive  insights.
\section{Underlying Motivation}
The rapid surge in digital data across all sectors necessitates innovative strategies for data analysis and interpretation. This escalating demand has been considerably addressed by machine learning, which leverages computational prowess to discover concealed patterns, make predictions, and aid in decision-making processes. However, despite progress, numerous challenges persist. These encompass issues such as the trustworthiness of crowdsourced labeling, efficient and reliable disease diagnosis, and driver distraction detection. Additionally, practical applications of machine learning are plagued with instances where data can contain missing, incorrect, or uncertain values, which often leads to unpredictable model behavior and inconsistent performance. It is no longer sufficient to solely consider model accuracy; accounting for uncertainty in model predictions is equally crucial. Armed with a comprehensive understanding of model and data uncertainty, we can make informed decisions on whether to rely on model predictions or seek further information. This research endeavor aims to tackle these challenges, augmenting the capabilities of machine learning and deepening our understanding of its potential.
\section{Problem Statement}
The core issue this dissertation addresses lies in the inherent complexity and variability seen in real-world datasets, and the challenges these present for machine learning applications. Specifically, it explores five separate issues, each related to a distinct domain. Despite their differences, a common thread unites these issues - the requirement for advanced, resilient, and precise computational methods capable of effectively handling and analyzing large, complex datasets. Machine learning, while promising, presents its own set of challenges unique to each application. These include issues such as the varying reliability of crowd annotators, the overlap of radiographic indications of various thoracic diseases, and the similar attributes of non-cilia elements and imaging noise in biological image analysis.
\section{Research Objectives}
This dissertation seeks to introduce and assess new machine learning methodologies for each of the problem areas, each aiming to enhance the accuracy, efficiency, and robustness of data interpretation in its corresponding field. By proposing these methods and evaluating their performance, it aims to enrich the understanding and development of machine learning applications. More specifically, the research objectives include creating an improved label aggregation technique that factors in uncertainty to provide a weighted aggregation scheme; devising accurate classification methods that leverage the taxonomic structure of different classes; developing robust techniques that use less common high contrast medical images to enhance the detection accuracy of more common lower contrast images; and establishing automated processes for detecting driver distraction and classifying primary cilia within microscopy images.
\section{Research Queries}
This dissertation seeks to answer several pivotal questions:
\begin{itemize}
    \item Can an innovative method increase the reliability of crowdsourced labeling and ensemble learning?
    \item How can machine learning tactics boost the precision of diagnosis by leveraging the taxonomic structure of medical pathologies?
    \item Is it possible to enhance the accuracy of less frequent data modalities by utilizing more common ones?
    \item How can transfer learning be utilized to effectively detect and classify driver distraction as well as primary cilia within microscopy images?
\end{itemize}
\section{Methodology Synopsis}
Each chapter of this dissertation proposes a unique methodology developed to tackle its respective problem, utilizing different machine learning techniques, including label aggregation, hierarchical multi-label classification, cascaded multi-planar schemes, and convolutional neural networks fused with random decision forests. The efficacy of each method is then assessed and compared with existing techniques, thereby providing a comprehensive understanding of their effectiveness.
\section{Scope and Limitations}
This dissertation delves into machine learning applications across a variety of domains. While the methods introduced can be generalized, their evaluations are carried out on specific datasets. Consequently, the findings and conclusions are subject to the constraints and properties of these datasets. Nonetheless, the proposed methods are scalable and adaptable, hinting at their wider application in the future. We cover a broad spectrum of machine learning applications, each addressing unique problems and methodologies. Although the primary focus is on the advancement of machine learning techniques, the impact of these methods stretches across diverse fields, including data science, radiology, road safety, and biology.
\section{Dissertation's Organization}
The structure of this dissertation is as follows.
\subsection*{Chapter~\ref{ch:crowd}: Crowd-Certain: Towards Robust Label Aggregation in Crowdsourced and Ensemble Learning Classification}
This chapter confronts the challenges of crowdsourced labeling by introducing a novel method for label aggregation termed as Crowd-Certain. This approach leverages the consistency of annotators versus a trained classifier to determine the reliability of each annotator, offering robustness and computational efficiency. The chapter delves into the workings of Crowd-Certain, discusses its core features, and evaluates its performance against ten other label aggregation techniques and for ten distinct datasets.
\subsection*{Chapter~\ref{ch:taxonomy}: Leveraging Disease Taxonomy for Enhanced Multi-Label Classification in Chest Radiography}
Chapter~\ref{ch:taxonomy} focuses on diagnosing thoracic diseases from chest radiographs using deep learning techniques. The chapter introduces two innovative hierarchical multi-label classification methods leveraging pathology taxonomy to enhance the accuracy and interpretability of disease classifications. The techniques cater to scenarios with and without available ground truth, broadening their adaptability. The chapter explores the methods, their evaluations on three large chest radiograph datasets (CheXpert~\cite{irvin_CheXpert_2019}, PadChest~\cite{bustos_Padchest_2020}, and NIH ChestXRay8~\cite{wang_ChestXRay8_2017}), and the proposed methods' potential benefits and limitations.
\subsection*{Chapter~\ref{ch:thalamus}: Automated Thalamic Nuclei Segmentation Using Multi-Planar Cascaded Convolutional Neural Networks}
This chapter develops a fast and accurate method for the segmentation of thalamic nuclei using a convolutional neural network (CNN) based approach. The proposed method uses a cascaded multi-planar scheme with a modified residual U-Net architecture. The novel approach delivers remarkable performance in speed, accuracy, and versatility. It also demonstrates its utility in studying thalamic nuclei atrophy in MS patients, providing potential for further advancements in neuroimaging.
\subsection*{Chapter~\ref{ch:drive-net}: Drive-Net: Convolutional Network for Driver Distraction Detection} This chapter presents Drive-Net, a novel supervised learning method that uses a CNN and a Random Decision Forest to classify images of drivers, effectively detecting driver distractions. Drive-Net's efficacy is validated through a comprehensive comparison with various other methods, achieving a higher detection accuracy rate. This chapter underlines the potential of automated methods in improving road safety measures.
\subsection*{Chapter~\ref{ch:cilia}: Classification of Primary Cilia in Microscopy Images Using Convolutional Neural Random Forests}
The final chapter utilizes the technique proposed in Chapter~\ref{ch:drive-net} to accurately detect and classify primary cilia in microscopy images. The classifier combines the feature learning abilities of CNNs with the practicality of decision trees, offering a unique solution to the task of cilia classification. The chapter details the workings of the classifier and compares its performance with traditional classifiers, demonstrating superior classification accuracy. This novel method showcases the potential intersection of computer science and biology in diagnosing primary ciliary dyskinesia.
