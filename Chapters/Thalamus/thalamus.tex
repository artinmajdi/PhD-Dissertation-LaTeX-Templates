% \documentclass{confproc}
\documentclass[3p,,final,12pt]{elsarticle}

\usepackage{tabulary,xcolor}
\usepackage{amsfonts,amsmath,amssymb}
\usepackage[T1]{fontenc}
\usepackage{ifluatex}
\ifluatex
\usepackage{fontspec}
\defaultfontfeatures{Ligatures=TeX}
\usepackage[]{unicode-math}
\unimathsetup{math-style=TeX}
\else 
\usepackage[utf8]{inputenc}
\usepackage{url,multirow,morefloats,floatflt,cancel,tfrupee}
\usepackage{textcomp}
\usepackage{colortbl}
\usepackage{xcolor}
\usepackage{pifont}
\usepackage[nointegrals]{wasysym}
% %%
%% Copyright 2007-2020 Elsevier Ltd
%%
%% This file is part of the 'Elsarticle Bundle'.
%% ---------------------------------------------
%%
%% It may be distributed under the conditions of the LaTeX Project Public
%% License, either version 1.2 of this license or (at your option) any
%% later version.  The latest version of this license is in
%%    http://www.latex-project.org/lppl.txt
%% and version 1.2 or later is part of all distributions of LaTeX
%% version 1999/12/01 or later.
%%
%% The list of all files belonging to the 'Elsarticle Bundle' is
%% given in the file `manifest.txt'.
%%
%% Template article for Elsevier's document class `elsarticle'
%% with harvard style bibliographic references



%% For including figures, graphicx.sty has been loaded in
%% elsarticle.cls. If you prefer to use the old commands
%% please give \usepackage{epsfig}

\makeatletter\if@twocolumn\PassOptionsToPackage{switch}{lineno}\else\fi\makeatother


\usepackage{tabulary,xcolor}
\usepackage{amsfonts,amsmath,amssymb}
\usepackage[T1]{fontenc}
\makeatletter
\let\save@ps@pprintTitle\ps@pprintTitle
\def\ps@pprintTitle{\save@ps@pprintTitle\gdef\@oddfoot{\footnotesize\itshape \null\hfill\today}}
\def\hlinewd#1{%
  \noalign{\ifnum0=`}\fi\hrule \@height #1%
  \futurelet\reserved@a\@xhline}
\def\tbltoprule{\hlinewd{.8pt}\\[-12pt]}
\def\tblbottomrule{\noalign{\vspace*{6pt}}\hline\noalign{\vspace*{2pt}}}
\def\tblmidrule{\noalign{\vspace*{6pt}}\hline\noalign{\vspace*{2pt}}}
\AtBeginDocument{\ifNAT@numbers \biboptions{sort&compress}\fi}
\makeatother

\usepackage{endfloat}

\usepackage{caption}
\captionsetup[figure]{labelsep=none,textformat=empty}
\usepackage{ifluatex}
\ifluatex
\usepackage{fontspec}
\defaultfontfeatures{Ligatures=TeX}
\usepackage[]{unicode-math}
\unimathsetup{math-style=TeX}
\else
\usepackage[utf8]{inputenc}
\fi
\ifluatex\else\usepackage{stmaryrd}\fi


%%%%%%%%%%%%%%%%%%%%%%%%%%%%%%%%%%%%%%%%%%%%%%%%%%%%%%%%%%%%%%%%%%%%%%%%%%
% Following additional macros are required to function some
% functions which are not available in the class used.
%%%%%%%%%%%%%%%%%%%%%%%%%%%%%%%%%%%%%%%%%%%%%%%%%%%%%%%%%%%%%%%%%%%%%%%%%%
\usepackage{url,multirow,morefloats,floatflt,cancel,tfrupee}
\makeatletter


\AtBeginDocument{\@ifpackageloaded{textcomp}{}{\usepackage{textcomp}}}
\makeatother
\usepackage{colortbl}
\usepackage{pifont}
\usepackage[nointegrals]{wasysym}
\urlstyle{rm}
\makeatletter

%%%For Table column width calculation.
\def\mcWidth#1{\csname TY@F#1\endcsname+\tabcolsep}

%%Hacking center and right align for table
\def\cAlignHack{\rightskip\@flushglue\leftskip\@flushglue\parindent\z@\parfillskip\z@skip}
\def\rAlignHack{\rightskip\z@skip\leftskip\@flushglue \parindent\z@\parfillskip\z@skip}

%Etal definition in references
\@ifundefined{etal}{\def\etal{\textit{et~al}}}{}


%\if@twocolumn\usepackage{dblfloatfix}\fi
\usepackage{ifxetex}
\ifxetex\else\if@twocolumn\@ifpackageloaded{stfloats}{}{\usepackage{dblfloatfix}}\fi\fi

\AtBeginDocument{
\expandafter\ifx\csname eqalign\endcsname\relax
\def\eqalign#1{\null\vcenter{\def\\{\cr}\openup\jot\m@th
  \ialign{\strut$\displaystyle{##}$\hfil&$\displaystyle{{}##}$\hfil
      \crcr#1\crcr}}\,}
\fi
}

%For fixing hardfail when unicode letters appear inside table with endfloat
\AtBeginDocument{%
  \@ifpackageloaded{endfloat}%
   {\renewcommand\efloat@iwrite[1]{\immediate\expandafter\protected@write\csname efloat@post#1\endcsname{}}}{\newif\ifefloat@tables}%
}%

\def\BreakURLText#1{\@tfor\brk@tempa:=#1\do{\brk@tempa\hskip0pt}}
\let\lt=<
\let\gt=>
\def\processVert{\ifmmode|\else\textbar\fi}
\let\processvert\processVert

\@ifundefined{subparagraph}{
\def\subparagraph{\@startsection{paragraph}{5}{2\parindent}{0ex plus 0.1ex minus 0.1ex}%
{0ex}{\normalfont\small\itshape}}%
}{}

% These are now gobbled, so won't appear in the PDF.
\newcommand\role[1]{\unskip}
\newcommand\aucollab[1]{\unskip}

\@ifundefined{tsGraphicsScaleX}{\gdef\tsGraphicsScaleX{1}}{}
\@ifundefined{tsGraphicsScaleY}{\gdef\tsGraphicsScaleY{.9}}{}
% To automatically resize figures to fit inside the text area
\def\checkGraphicsWidth{\ifdim\Gin@nat@width>\linewidth
	\tsGraphicsScaleX\linewidth\else\Gin@nat@width\fi}

\def\checkGraphicsHeight{\ifdim\Gin@nat@height>.9\textheight
	\tsGraphicsScaleY\textheight\else\Gin@nat@height\fi}

\def\fixFloatSize#1{}%\@ifundefined{processdelayedfloats}{\setbox0=\hbox{\includegraphics{#1}}\ifnum\wd0<\columnwidth\relax\renewenvironment{figure*}{\begin{figure}}{\end{figure}}\fi}{}}
\let\ts@includegraphics\includegraphics

\def\inlinegraphic[#1]#2{{\edef\@tempa{#1}\edef\baseline@shift{\ifx\@tempa\@empty0\else#1\fi}\edef\tempZ{\the\numexpr(\numexpr(\baseline@shift*\f@size/100))}\protect\raisebox{\tempZ pt}{\ts@includegraphics{#2}}}}

%\renewcommand{\includegraphics}[1]{\ts@includegraphics[width=\checkGraphicsWidth]{#1}}
\AtBeginDocument{\def\includegraphics{\@ifnextchar[{\ts@includegraphics}{\ts@includegraphics[width=\checkGraphicsWidth,height=\checkGraphicsHeight,keepaspectratio]}}}

\DeclareMathAlphabet{\mathpzc}{OT1}{pzc}{m}{it}

\def\URL#1#2{\@ifundefined{href}{#2}{\href{#1}{#2}}}

%%For url break
\def\UrlOrds{\do\*\do\-\do\~\do\'\do\"\do\-}%
\g@addto@macro{\UrlBreaks}{\UrlOrds}



\edef\fntEncoding{\f@encoding}
\def\EUoneEnc{EU1}
\makeatother
\def\floatpagefraction{0.8}
\def\dblfloatpagefraction{0.8}
\def\style#1#2{#2}
\def\xxxguillemotleft{\fontencoding{T1}\selectfont\guillemotleft}
\def\xxxguillemotright{\fontencoding{T1}\selectfont\guillemotright}

\newif\ifmultipleabstract\multipleabstractfalse%
\newenvironment{typesetAbstractGroup}{}{}%

%%%%%%%%%%%%%%%%%%%%%%%%%%%%%%%%%%%%%%%%%%%%%%%%%%%%%%%%%%%%%%%%%%%%%%%%%%
\emergencystretch 20pt \tolerance = 1500 \def\floatpagefraction{0.8}




\makeatletter
\def\ead{\@ifnextchar[{\@uad}{\@ead}}
\gdef\@ead#1{\bgroup
\def\_{\string\underscorechar\space}
\def\{{\string\lbracechar\space}
\def\textdagger{\string\textdagger\space}
\def\texttildeapprox{\string\texttildeapprox\space}
\def~{\hashchar\space}
\def\}{\string\rbracechar\space}
\edef\tmp{\the\@eadauthor}
\immediate\write\@auxout{\string\emailauthor {#1}{\expandafter\strip@prefix\meaning\tmp}}
\egroup
}
%\newcounter{ead}
\gdef\emailauthor#1#2{\stepcounter{ead}
\g@addto@macro\@elseads{\raggedright
\let\corref\@gobble
\eadsep\texttt{#1} (#2)
\def\eadsep{\unskip,\space}}
}

\makeatother

\usepackage{float}

\journal{Nuclear Physics B}


\definecolor{mygreen}{HTML}{E9ECE6}


\begin{document}

\title{Automated Thalamic Nuclei Segmentation Using Multiplanar Cascaded Convolutional Neural Networks}

\begin{abstract}
\textbf{Purpose:} To develop a fast and accurate convolutional neural network based method for segmentation of thalamic nuclei.

\textbf{Conclusion:} The proposed segmentation method is fast, accurate, performs well across disease types and field strengths, and shows great potential to improve our understanding of the role of thalamic nuclei in neurological diseases. 

\textbf{Methods:} A cascaded multiplanar scheme with a modified residual U-Net architecture was used to segment the thalamic nuclei in conventional and white-matter-nulled magnetization-prepared rapid gradient echo (MPRAGE) data. A single network was optimized to work with images from healthy controls and patients with multiple sclerosis (MS) and essential tremor (ET), acquired in 3T and 7T field strengths. WMn-MPRAGE images were manually delineated by a trained neuroradiologist using the Morel histological atlas as a guide to generate reference ground-truth labels. Dice similarity coefficient and volume similarity index (VSI) were used to evaluate performance. The clinical utility of applying this method to study the effect of MS on thalamic nuclei atrophy was demonstrated. 

\textbf{Results:} Segmentation of each thalamus into 12 nuclei was achieved in less than a minute. For 7T WMn-MPRAGE, the proposed method outperforms the current state of the art in patients with ET with statistically significant improvements in Dice for five nuclei (increase in the range of 0.05-0.18) and VSI for four nuclei (increase in the range of 0.05-0.19), while performing comparably for healthy subjects and MS subjects. Dice and VSI achieved using 7T WMn-MPRAGE data are comparable to those using 3T WMn-MPRAGE data. For conventional MPRAGE, the proposed method shows a statistically significant Dice improvement in the range of 0.14-0.63 over FreeSurfer for all nuclei and disease types. The effect of noise on network performance shows robustness to images with SNR as low as half the baseline SNR. Atrophy of four thalamic nuclei and the entire thalamus was observed in MS patients compared to healthy control subjects, after controlling for the effect of parallel imaging, intracranial volume, gender, and age (p{\textless}0.004).

\textbf{KEYWORDS:} Convolutional neural network, Thalamic nuclei segmentation, Clinical analysis, White-matter-nulled MPRAGE
\end{abstract}

\maketitle

    
\section{Introduction}
 The thalamus is a deep brain gray matter structure that relays information between various subcortical areas and the cerebral cortex \unskip~\cite{1643371:26789934} and plays a critical role in the regulation of sleep, consciousness, arousal, and awareness \unskip~\cite{1643371:26789969,1643371:26789903,1643371:26789938}. It is subdivided into multiple nuclei with varying functions. Thalamic involvement has been reported in schizophrenia \unskip~\cite{1643371:26789949,1643371:26789927}, alcohol use disorder (AUD) \unskip~\cite{1643371:26789925,1643371:26789936}, Parkinson's disease \unskip~\cite{1643371:26789951}, multiple sclerosis (MS) \unskip~\cite{1643371:26789909}, essential tremor \unskip~\cite{1643371:26789937,1643371:26789916,1643371:26789943} and Alzheimer's disease \unskip~\cite{1643371:26789944}. These pathologies affect different thalamic nuclei differently and, therefore, accurate volumetry of the thalamic nuclei can be beneficial for tracking disease progression and treatment efficacy \unskip~\cite{1643371:26789944,1643371:26789946}. 

Manual delineation of thalamic nuclei from in-vivo scans is very tedious and requires specialized knowledge \unskip~\cite{1643371:26789922,1643371:26789924}.\ensuremath{^{}}Due to low intra-thalamic contrast \unskip~\cite{1643371:26789961}, thalamic nuclei are not easily distinguishable in conventional T1- and T2-weighted magnetic resonance images. As a result, most automated methods based on structural MRI have only segmented the entire thalamus as part of subcortical brain segmentation \unskip~\cite{1643371:26789935,1643371:26789929,1643371:26789926,1643371:26789933,1643371:26789917}.\ensuremath{^{}}

Diffusion tensor imaging (DTI) based methods that use either local or global properties of the diffusion tensor have been more popular for thalamic nuclei segmentation. Behrens\unskip~\cite{1643371:26789967} used tractography of cortical projections to the thalamus to segment the thalamic regions, but this method requires precise knowledge of neuroanatomy to identify the relevant cortical regions. More automated and computationally efficient methods have been proposed that use k-means clustering of the dominant diffusion orientation to achieve thalamic parcellation \unskip~\cite{1643371:26789911,1643371:26789930,1643371:26789941}.\ensuremath{^{}}To date, the most consistent DTI-based method \unskip~\cite{1643371:26789918}\ensuremath{^{}} uses spherical harmonic decomposition-based orientation distribution functions to achieve robust segmentation of seven thalamic nuclei. However, the low spatial resolution of echo-planar imaging, which underlies DTI, and the predominance of gray matter in the thalamus, which results in low anisotropy, make these DTI-based methods suboptimal \unskip~\cite{1643371:26789960}, often resulting in segmentation of only the larger thalamic groups. Advanced techniques such as susceptibility-weighted imaging (SWI) \unskip~\cite{1643371:26789920} can provide better intra-thalamic contrast and have been used for segmentation of thalamic nuclei at 7T \unskip~\cite{1643371:26789968,1643371:26789923}. However, these methods have found limited application in presurgical targeting, focusing mainly on the VIM nucleus. Hybrid approaches that combine DTI with T1 weighted imaging have also been proposed \unskip~\cite{1643371:26789958,1643371:26789948}.

Recently, high-spatial-resolution structural MRI has been investigated for thalamic nuclei segmentation. The most widely used T1-weighted structural MRI sequence is magnetization prepared rapid gradient echo (MPRAGE), where the cerebrospinal fluid (CSF) is nulled. We refer to this method as CSFn-MPRAGE. Iglesias\unskip~\cite{1643371:26789917} proposed a probabilistic atlas constructed using manual delineation of 26 thalamic nuclei per thalamus in six autopsy samples and used Bayesian inference to segment 3T MPRAGE images into 26 nuclei per side \unskip~\cite{1643371:26789932,1643371:26789914}. However, this method is very time-consuming, requiring multiple hours for the segmentation of one subject, and has not been thoroughly validated against manual segmentation \unskip~\cite{1643371:26789917}. Liu \unskip~\cite{1643371:26789952} segmented thalamic nuclei from 3T T1-weighted MRI data using an atlas developed from multiple MPRAGE and SWI sequences acquired at 7T. A multi-atlas label fusion and statistical shape modeling algorithm were used to transfer from 7T to 3T. Variants of the MPRAGE sequence have been proposed to better visualize the intra-thalamic structures \unskip~\cite{1643371:26789957,1643371:26789954}. Su \unskip~\cite{1643371:26789960} used a WMn-MPRAGE sequence that is optimized for intra-thalamic contrast \unskip~\cite{1643371:26789961} in conjunction with a multiatlas technique, called "thalamus optimized multiatlas segmentation" (THOMAS) to segment the thalamus into 12 nuclei. The performance of this based method depends on the accuracy of a computationally expensive registration step \unskip~\cite{1643371:26789917,1643371:26789939}.\ensuremath{^{}}This method has only been validated on specialized WMn-MPRAGE data.

Convolutional neural networks (CNNs) are a class of deep learning techniques that use convolutional kernels to capture the nonlinear mapping between an input image and its segmentation labels. Unlike atlas-based segmentation techniques, CNNs do not depend on image registration and manual feature extraction \unskip~\cite{1643371:26789908}. Although many studies have explored the advantages of using CNNs for subcortical segmentation \unskip~\cite{1643371:26789955,1643371:26789913,1643371:26789947,1643371:26789902}, these studies are limited to the entire thalamus. Due to the paucity of training data and the high computational and memory requirements of 3D analysis, most proposed methods make use of 2D CNNs \unskip~\cite{1643371:26789947}. However, the use of 2D networks does not fully exploit the anatomical information present in the 3D MRI data. Alternatively, multiplanar techniques that make use of 2D CNNs along the three orthogonal planes have been shown \unskip~\cite{1643371:26789913,1643371:26789921,1643371:26789904,1643371:26789950}\ensuremath{^{}}to improve segmentation performance with lower computational cost than a full 3D analysis. Transfer learning techniques have also been investigated to mitigate the lack of sufficient training data \unskip~\cite{1643371:26789942}.

In this work, we propose the use of a modified residual U-Net in a cascaded multiplanar scheme for thalamic nuclei segmentation. We first demonstrate this method for WMn-MPRAGE \unskip~\cite{1643371:26789961} data, evaluating it on data from 7T and 3T. We then extend this work to the more commonly acquired CSFn-MPRAGE data by fine-tuning the network trained on WMn-MPRAGE data. The performance of both networks is validated in healthy subjects and patients with MS and ET and compared to current state-of-the-art segmentation methods. Finally, the robustness of the proposed method to SNR and its applicability to data from patients with multiple sclerosis are investigated.
    
\section{METHODS}




\subsection{Network Architecture} The proposed method uses a modified residual U-Net architecture (mRU-Net) for 2D segmentation of the thalamic nuclei, as shown in Figure~\ref{f-49677889c7e1}  (top panel). The conventional U-Net \unskip~\cite{1643371:26789965} uses a contracting (encoder) layer and an expanding (decoder) layer for multi-resolution feature extraction and synthesis. To improve convergence performance and reduce overfitting, we incorporated batch normalization \unskip~\cite{1643371:26789970} and dropout \unskip~\cite{1643371:26789907} layers into the network. Residual convolutional blocks \unskip~\cite{1643371:26789905} were included to mitigate the problem of vanishing and exploding loss function gradients. Each convolutional block consists of 2D convolution units followed by batch normalization and a leaky rectified linear unit \unskip~\cite{1643371:26789910} (leakage factor 0.1). A 1x1 convolution with a sigmoid activation function maps the final feature maps into the desired number of classes, generating a probability map for each class. 

For thalamic segmentation, we use a two-step cascaded approach \unskip~\cite{1643371:26789931}, as shown in Figure~\ref{f-49677889c7e1}  (bottom panel). A first mRU-Net initially segments the whole thalamus, and this guides a second mRU-Net, which segments the different thalamic nuclei. Preprocessed 2D images (see Section 2.3 for details) along with their manual segmentation masks are used to train the first network to segment the whole thalamus. A bounding box that encompasses the entire thalamus is used to crop the input image (and its corresponding manual segmentation masks) before it is input to the second network to perform thalamic nuclei segmentation. Note that an optional preprocessing cropping step (``Cropping 1'' in Figure~\ref{f-49677889c7e1} ) is also added before inputting images into the first network.

\begin{figure*}[!htbp]
\centering 
\includegraphics[width=\textwidth]{Thalamus/images/slide1.jpeg}
\caption{The top panel shows the mRU-Net architecture, where normalization and residual blocks are added to a standard U-Net. The bottom panel shows the proposed cascaded network for segmentation of the thalamic nuclei. Following preprocessing, mRU-Net1 is trained to segment the whole thalamus. The output of this network is used to find a bounding box (red square) that encompasses the left thalamus. The cropped inputs are used to train a second network (mRU-Net2) to segment the thalamic nuclei. }
\label{f-49677889c7e1}
\end{figure*}



Due to the limitation in the number of labeled datasets available, training a generalizable 3D CNN that can exploit the 3D structure is infeasible. To overcome this limitation and take advantage of the spatial information present in an isotropic resolution 3D MRI dataset, we use a multi-planar \unskip~\cite{1643371:26789950} approach as shown in Figure~\ref{f-afc126ad9a6a} . The input 3D dataset is preprocessed, reformatted into three orthogonal orientations (axial, coronal, and sagittal), and the resulting 2D images are fed into three cascaded 2D networks (from Figure~\ref{f-49677889c7e1} ) to take advantage of the complementary information from each orientation. For each cascaded network, the output segmentations are reformatted to the original imaging orientation and then fused using voxel-wise majority voting.


\begin{figure*}[!htbp]
\centering \includegraphics{Thalamus/images/slide2.jpeg}
\caption{Schematic of the multiplanar scheme, based on the cascaded network of Figure~\ref{f-49677889c7e1} . The 3D input image is reformatted to the sagittal, coronal, and axial planes and fed as 2D inputs to three separate cascaded networks. The segmentations from the three 2D networks are fused using majority voting to generate the final 3D thalamic nuclei segmentation. }
\label{f-afc126ad9a6a}
\end{figure*}

In the bottom panel of Figure 1, the first network (whole thalamus segmentation) uses a Dice \unskip~\cite{1643371:26789956} loss function along with sigmoid activation. For the second network, we investigated two different loss functions with a one-hot encoding approach in treating multiple classes: a) Weighted cross-entropy (Equation~(\ref{dfg-8128ef3038a4}) ) and b) Dice (Equation~(\ref{disp-formula-group-2aaeb239840b4f1cb28b6c85d1a9f87b}) ).

Two instances of the cascaded multiplanar network were implemented: a WMn network for segmenting WMn-MPRAGE images and a CSFn network for segmenting CSFn-MPRAGE images.



\subsection{Datasets and Labels} To evaluate the segmentation of the thalamic nuclei of the WMn-MPRAGE images, we used data from 40 subjects (13 healthy subjects, 15 MS patients and 12 ET patients) acquired on a 7T scanner. We used data from the 12 ET patients rescanned on a 3T scanner to extend this work to 3T. To evaluate thalamic nuclei segmentation of CSFn-MPRAGE images, we used data from 33 subjects (6 healthy subjects, 15 patients with MS, and 12 patients with ET) scanned with both WMn-MPRAGE and CSFn-MPRAGE pulse sequences on a 7T scanner. A separate dataset consisting of 93 WMn-MPRAGE images from a 3T scanner was used just for the purpose of network initialization (referred to as the initialization dataset); this included patients with alcohol use disorder (AUD) and healthy control subjects. Note that data from AUD patients (whose images are similar to healthy subjects except for slight atrophy \unskip~\cite{1643371:26789962}) were used solely to obtain initialization weights for the WMn network to help accelerate network convergence. All subjects were scanned on GE (General Electric Healthcare, Waukesha, WI) 7T and 3T scanners after obtaining prior informed consent and adhering to institutional review board protocols. The pulse sequence parameters for both the WMn-MPRAGE and CSFn-MPRAGE pulse sequences are listed in Table~\ref{tw-3be63685482d}.


\begin{table*}[!htbp]
\caption{Image acquisition parameters for WMn- and CSFn-MPRAGE for 7T and 3T data used in this study}
\label{tw-3be63685482d}
\def\arraystretch{1}
\ignorespaces 
\centering 
\begin{tabulary}{\linewidth}{LLLL}
\tbltoprule 
 &
  \multicolumn{2}{p{\dimexpr(\mcWidth{2}+\mcWidth{3})}}{WMn-MPRAGE} &
  CSFn-MPRAGE\\
Scanner &
  3T (GE) &
  7T (GE) &
  7T (GE)\\
TR/TS (ms) &
  10/4500 &
  10/6000 &
  7.2/3000\\
TI (ms) &
  500 &
  680 &
  1200\\
Flip (deg) &
  9 &
  4 &
  6\\
Matrix  &
  180 \ensuremath{\times} 220 \ensuremath{\times} 180 &
  180 \ensuremath{\times} 220 \ensuremath{\times} 180 &
  180 \ensuremath{\times} 220 \ensuremath{\times} 180\\
Acq. resolution (mm) &
  1x1x1 &
  1x1x1 &
  1x1x1\\
Rec. resolution (mm) &
  0.7x0.5x0.7 &
  0.7x0.5x0.7 &
  0.7x0.5x0.7\\
Parallel imaging &
  None &
  1.5 \ensuremath{\times} 1.5 &
  3x1\\
Coil &
  8-channel GE &
  32-channel Nova &
  32-channel Nova\\
\tblbottomrule 
\end{tabulary} \par 
\end{table*}

The reference labels for the WMn-MPRAGE images were generated using manual segmentation of the thalamic nuclei, performed by a trained neuroradiologist using the Morel histological atlas \unskip~\cite{1643371:26789963} as a guide. Eleven thalamic nuclei, the whole thalamus, and the mammillothalamic tract (MTT) were delineated using freehand spline drawing tools to build the 3D vector-based model of each structure. 

The eleven delineated nuclei are grouped as follows:



\begin{itemize}
  \item \relax \textbf{medial group}: mediodorsal (MD), centromedian (CM), habenula (Hb)
  \item \relax \textbf{posterior group}: pulvinar (Pul), medial geniculate nucleus (MGN), lateral geniculate nucleus (LGN)
  \item \relax \textbf{lateral group}: ventral posterolateral (VPL), ventral lateral anterior (VLa), ventral lateral posterior (VLp), ventral anterior nucleus (VA)
  \item \relax \textbf{anterior group}:  anteroventral (AV)
\end{itemize}
  Very small nuclei, such as ventral posteromedial, centrolateral, and intralaminar nuclei, which are part of the Morel atlas could not reliably visualized and segmented by the neuroradiologist and were, hence, omitted. As a result, there are lacunae in the gold standard reference labels. Reference labels for CSFn-MPRAGE images were obtained by affine registering WMn-MPRAGE images to the corresponding CSFn-MPRAGE images for each subject and warping the labels using nearest-neighbor interpolation. Reference labels for the initialization data set were obtained using THOMAS segmentation \unskip~\cite{1643371:26789960}. Due to the laborious nature of manual segmentation, only the left thalamic nuclei were manually segmented. As a result, all networks were trained on thely on left thalamic nuclei. To delineate the right thalamic nuclei, the input images were flipped in the left-right direction and segmented using the networks trained on the left manual segmentation labels. Following the segmentation of the flipped images, both the images and the predicted labels were flipped back to their original orientation.



\subsection{Network Training} All networks were implemented in Python and Keras \unskip~\cite{1643371:26789919} with a TensorFlow backend using an NVIDIA P100 GPU with 16 GB of GDDR5 RAM (source code available at \BreakURLText{https://github.com/artinmajdi/Thalamic-Nuclei-Segmentation}). The WMn network was trained using combined 3T and 7T data. The CSFn network was initialized using this WMn network and then fine-tuned (i.e. retrained using WMn labels registered to the CSFn data) to adapt to CSFn-MPRAGE contrast. The networks were trained using Adam optimizer \unskip~\cite{1643371:26789915} with 300 epochs and a batch size of 100. Network performance was evaluated for both weighted cross entropy and Dice loss functions. The number of layers, the number of convolutional feature maps, and the learning rates were chosen on the basis of hyperparameter tuning. Furthermore, a scheduler was used to set the learning rate in each epoch, starting from an initial value of 0.001 that reduces by a factor of 0.5 if the validation Dice plateaus for 15 epochs. In the multiplanar approach, after a series of hyperparameter tunings, the number of feature maps in the first layer was set to 40, 30, and 20 for the sagittal, coronal, and axial networks, respectively. The number of feature maps in each subsequent layer was increased by a factor of two, as proposed in conventional U-Net \unskip~\cite{1643371:26789965}. For the WMn and CSFn networks, 20\% and 25\% of data were randomly selected for validation, while the remaining 80\% and 75\% were used for training, respectively. 



\subsection{Pre- and Post-Processing}All input images were preprocessed by performing N4 bias field correction and zero mean unit standard deviation normalization. They were then reformatted to a standard imaging plane and resolution (axial with 0.7 x 0.5 x 0.7 mm\ensuremath{^{3}}). To ensure enough variability, the training data were augmented using random in-plane rotations of up to $\pm7^\circ $in three different planes, creating six new images in each orientation and a net increase in the number of training data by a factor of 18. To reduce the memory and computation power requirements, a cropping step was added to the preprocessing as an additional option. To automate this cropping step, we used a WMn-MPRAGE template created from mutual registration and the averaging of 20 prior WMn-MPRAGE datasets as described in Su \unskip~\cite{1643371:26789960}. A bounding box covering both thalami was manually drawn once, on this template and warped into the input image space by affine registering the template with the input image. The preprocessing steps are shown in Figure~\ref{f-49677889c7e1}  (bottom panel). Typical input sizes for the original, preprocessed (input to the first whole thalamus network) and cropped images (input to the second nuclei segmentation network) were 256x373x256, 97x94x63, and 52x84x48, respectively.



\subsection{Performance Evaluation Measures}The automated segmentation performance was evaluated with respect to manual delineations using the Dice similarity coefficient \unskip~\cite{1643371:26789928}, and volume similarity index (VSI) \unskip~\cite{1643371:26789960}. 

Dice loss function is defined as follows:
\let\saveeqnno\theequation
\let\savefrac\frac
\def\dispfrac{\displaystyle\savefrac}
\begin{eqnarray}
\let\frac\dispfrac
\gdef\theequation{1}
\let\theHequation\theequation
\label{disp-formula-group-f73e8e30e04042bea52de88dd1c0e353}
\begin{array}{@{}l}\text{Loss}=\frac1C{\sum\nolimits_{i=0}^{C-1}{(1-\;\text{Dice}\left(A_i,B_i)\right)}}\end{array}
\end{eqnarray}
\global\let\theequation\saveeqnno
\addtocounter{equation}{-1}\ignorespaces 

\let\saveeqnno\theequation
\let\savefrac\frac
\def\dispfrac{\displaystyle\savefrac}
\begin{eqnarray}
\let\frac\dispfrac
\gdef\theequation{2}
\let\theHequation\theequation
\label{disp-formula-group-2aaeb239840b4f1cb28b6c85d1a9f87b}
\begin{array}{@{}l}\text{Dice}\left(A_i,B_i\right)=\frac{2\left\vert A_i\cap B_i\right\vert}{\left\vert A_i\cup\;B_i\right\vert}\end{array}
\end{eqnarray}
\global\let\theequation\saveeqnno
\addtocounter{equation}{-1}\ignorespaces 
where $A_i $ and $B_i $ are automated and manual segmentations for the nucleus $i $ and $\vert\;.\;\vert $ is the cardinality of a set (i.e., number of pixels). Note that the different classes are implicitly weighted when using a Dice-based loss function \unskip~\cite{1643371:26789964,1643371:26789945}.\ensuremath{^{}}A weighted cross entropy was used to mitigate the effect of class imbalance, where the weight for each class is defined as the ratio of the number of foreground to background voxels in the manual segmentation mask. The weighted cross-entropy loss is defined as
\let\saveeqnno\theequation
\let\savefrac\frac
\def\dispfrac{\displaystyle\savefrac}
\begin{eqnarray}
\let\frac\dispfrac
\gdef\theequation{3}
\let\theHequation\theequation
\label{dfg-8128ef3038a4}
\begin{array}{@{}l}\text{Loss}=\;-\sum\nolimits_{i=0}^{C-1}w_i\left(q_i\;\text{log}\left(p_i\right)+\left(1-q_i\right)\;\text{log}\left(1-p_i\right)\right)\;\;\;\;\;\end{array}
\end{eqnarray}
\global\let\theequation\saveeqnno
\addtocounter{equation}{-1}\ignorespaces 
where $C $ is the number of classes that include the background (2 for the entire thalamus network or 13 for the subsequent thalamic nuclei segmentation network for 12 nuclei), $p_i={{exp}{({\widehat y}_i)}}/{\sum\nolimits_{k=0}^{C-1}{{{exp}{({\widehat y}_k)}}\;}} $is the posterior probability of class $i $ obtained by applying the sigmoid function to the final features maps of the network ${\widehat y}_i $, $w_i $ represents the weight for class $i $, and $q_i $ is the ground truth for class $i $.

The Dice similarity coefficient (shortened as Dice) shown in Equation~(\ref{disp-formula-group-2aaeb239840b4f1cb28b6c85d1a9f87b})  measures overlap between the manual segmentation and the automated segmentation computed by the network. The volume similarity index is defined as 
\let\saveeqnno\theequation
\let\savefrac\frac
\def\dispfrac{\displaystyle\savefrac}
\begin{eqnarray}
\let\frac\dispfrac
\gdef\theequation{4}
\let\theHequation\theequation
\label{disp-formula-group-2a067b4f3fd04fa9843b3e7ae057d0e6}
\begin{array}{@{}l}\text{VSI}(A,B)\;=\;\frac1C\sum\nolimits_{i=0}^{C-1}\left(1-\frac{\left\vert \left\vert A_i\right\vert-\left\vert B_i\right\vert\right\vert}{\left\vert A_i\right\vert+\left\vert B_i\right\vert}\right)\end{array}
\end{eqnarray}
\global\let\theequation\saveeqnno
\addtocounter{equation}{-1}\ignorespaces 
where $A_i $ and $B_i $ are the automated and manual segmentations for nucleus $i $, with a VSI of 1 indicating identical volumes. 

Network outputs were binarized using thresholds computed from precision-recall curves. The threshold (0.7) was determined by finding the trade-off between the precision and recall values that would correspond to the lowest false positive and false negative rates. 



\subsection{Experiments}



\subsubsection{Network Optimization} To potentially reduce memory and computational burden, we explored an additional step to crop input images from the first thalamus segmentation network (``Cropping 1'' in the lower panel of Figure~\ref{f-49677889c7e1}). The performance of the cascaded scheme was also compared to a non-cascaded scheme, which is a network similar to the proposed cascaded scheme with the exception of removing the second network's cropping step (``Cropping 2" in Figure~\ref{f-49677889c7e1}). Finally, the effect of network initialization was studied by training two separate networks, one using random initialization and the other initialized using weights from a separate network that was trained on 3T WMn-MPRAGE data with THOMAS reference labels (3T initialization network). Lastly, to find the optimal loss function, two networks were trained separately using weighted binary cross-entropy and Dice loss functions, respectively. All the above experiments were carried out in a subset (n = 11) of subjects randomly chosen from the WMn-MPRAGE data set and distributed equally among control subjects and disease types, except for cropping experiments which were carried out on the entire dataset (n=40).



\subsubsection{Network Performance} To evaluate the segmentation performance of the optimized multiplanar cascaded networks, a four-fold and eight-fold cross-validation was performed using approximately 20\% (out of 52 cases) and 25\% (out of 33 cases) for each fold in the WMn-MPRAGE and CSFn-MPRAGE data, respectively. Data in cross-validation folds were equitably distributed with respect to control/disease type. To extend the applicability of the CNN-based method to 3T, data from 12 ET patients rescanned on a 3T scanner were used as part of the cross-validation along with all 7T WMn-MPRAGE data. The segmentation results of our method were compared to the THOMAS and FreeSurfer-based segmentations for WMn-MPRAGE and CSFn-MPRAGE images, respectively, using Dice and VSI measures. 



\subsection{Network Robustness to Noise}Many factors influence the signal-to-noise ratio (SNR) of the MR image, including the resistance of the receiver coil, inductive losses in the sample \unskip~\cite{1643371:26789940}, image voxel size, receiver bandwidth \unskip~\cite{1643371:26789953}, and pulse sequence parameters. To study the robustness of our method to different levels of SNR, randomly generated noise was incorporated using Equation~(\ref{disp-formula-group-b190c3da78ae41a388c05abb6b380372})  into a WMn-MPRAGE magnitude image I~(resulting in a Rician noise corrupted image $I' $ ). 
\let\saveeqnno\theequation
\let\savefrac\frac
\def\dispfrac{\displaystyle\savefrac}
\begin{eqnarray}
\let\frac\dispfrac
\gdef\theequation{5}
\let\theHequation\theequation
\label{disp-formula-group-b190c3da78ae41a388c05abb6b380372}
\begin{array}{@{}l}I^{'}=\sqrt{{\left(I+n_{\text{real}}\right)}^{2}+n_{\text{imag}}^{2}}\end{array}
\end{eqnarray}
\global\let\theequation\saveeqnno
\addtocounter{equation}{-1}\ignorespaces 
where $n_{\text{real}} $ and $n_{\text{imag}} $ are independent Gaussian distributed random variables $N(0,\sigma ^{2}) $,. Thalamic SNR is measured as the ratio of the mean thalamic signal to the standard deviation of the noise from two regions of interest (ROIs) placed over the thalamus and the image background. Starting from the original acquired WMn-MPRAGE images (nominal thalamic SNR of 23.5), noise with increasing standard deviation $\sigma $ was added to produce 10 images with an SNR in the range of 23.5-8 (i.e. {\texttildeapprox} 3X degradation in thalamic SNR).



\subsection{Clinical Analysis in Multiple Sclerosis} To assess the effect of multiple sclerosis on specific thalamic nuclei, a covariance analysis (ANCOVA) was performed to compare the volumes of the thalamic nuclei between healthy controls and MS patients, controlling for age, sex, parallel imaging (PI) and intracranial cavity volume (ICV). The estimated total intracranial volume (eTIV) of the FreeSurfer segmentation was used for the ICV to account for different head sizes. The left and right thalamic nuclei were segmented from 7T WMn-MPRAGE data acquired on 15 MS patients (13 patients had relapsing remitting MS, while 2 patients had secondary progressive MS) and 13 healthy subjects (free of neurologic, psychiatric, or systemic diseases and drug or alcohol abuse). A Bonferroni corrected p-value of 0.05/11 = 0.005 (to account for multiple comparisons of the 11 nuclei) was used.
    
\section{RESULTS}




\subsection{Network Optimization}



\subsubsection{Training Time} The training time required on an NVIDIA P100 GPU card for the whole thalamus and multiclass thalamic nuclei segmentation was 1 hour and 1.5 hours, respectively, for a single imaging orientation. The cumulative training time for the multiplanar cascade scheme was $(1\;+\;1.5)\;\times\;3\;=\;7.5\; $ hours (accounting for the axial, coronal, and sagittal orientations). This number can be reduced to 2.5 hours if the training in each orientation is performed in parallel (three GPUs, one for each orientation). The time required for preprocessing and segmentation of each subject in the testing phase of the final multiplanar scheme was 3 min and 1 min, respectively. The use of the cascaded scheme reduced the memory required for training by up to 86\%, allowing the use of augmented data during the training process. Although the cascaded scheme was trained in the presence of $15x $ augmented data, the number of epochs and overall convergence time during training were reduced by 66\% and 21\%, respectively, compared to the non-cascaded algorithm. 



\subsubsection{Cropping, Initialization, Loss Function }For the 40 test subjects (13 control, 15 MS, 12 subjects with ET), no statistical differences were observed in the mean Dice between using uncropped and cropped input images to the thalamic segmentation network (Figure~\ref{f-49677889c7e1} ) while a 93\% reduction in memory requirements was achieved using the initial cropping step (``Cropping 1'' in Figure~\ref{f-49677889c7e1}). Following this experiment, all remaining experiments included this initial cropping step as part of preprocessing. 

The WMn network initialized using weights from a 3T initialization network showed a significant improvement in Dice for two nuclei (VA and Hb) and an increase in convergence rate compared to random initialization (Figure~\ref{f-67347f0d82dc} ). Following this, all experiments on WMn-MPRAGE data were performed using this initializing step. For the WMn network, the use of the Dice loss function showed a statistically significant improvement in Dice for the whole thalamus and 4 nuclei (VPl, LGN, CM, MGN) compared to a weighted cross entropy loss function (Table~\ref{tw-d60347b9dec1} ). Furthermore, it reduced the overall number of epochs, the required training time (per epoch), and the convergence time by 32\%, 16\%, and 43\%, respectively. As a result, the Dice loss function was used for all further experiments. 


\begin{table*}[!htbp]
\caption{Effect of loss function on Dice and VSI. Average Dice and VSI shown for 11 subjects.}
\label{tw-d60347b9dec1}
\def\arraystretch{1}
\ignorespaces 
\centering 
\begin{tabulary}{\linewidth}{LLLLL}
\tbltoprule 
 &
  \multicolumn{2}{p{\dimexpr(\mcWidth{2}+\mcWidth{3})}}{Dice} &
  \multicolumn{2}{p{\dimexpr(\mcWidth{4}+\mcWidth{5})}}{VSI}\\
Nuclei &
  Dice Loss &
  BCE Loss &
  Dice Loss &
  BCE Loss\\
Thalamus &
  0.92\textdagger   &
  0.91 &
  0.98\textdagger   &
  0.96\\
Pul  &
  0.86 &
  0.85 &
  0.97 &
  0.96\\
VLP  &
  0.78 &
  0.79 &
  0.95 &
  0.95\\
MD-Pf  &
  0.85 &
  0.85 &
  0.95 &
  0.94\\
VPl  &
  0.64\textdagger   &
  0.61 &
  0.92 &
  0.89\\
VA  &
  0.70 &
  0.69 &
  0.92 &
  0.93\\
AV   &
  0.76  &
  0.74  &
  0.87  &
  0.85 \\
VLa   &
  0.69  &
  0.67  &
  0.92  &
  0.89 \\
CM   &
  0.70\textdagger   &
  0.65  &
  0.93\textdagger   &
  0.86 \\
LGN   &
  0.70\textdagger   &
  0.66  &
  0.90  &
  0.85 \\
MGN   &
  0.70\textdagger   &
  0.64  &
  0.89  &
  0.81 \\
MTT   &
  0.68  &
  0.63  &
  0.88  &
  0.81 \\
Hb   &
  0.77  &
  0.75  &
  0.89  &
  0.88 \\
\tblbottomrule 
\end{tabulary}\par 
\end{table*}

\bgroup
\fixFloatSize{images/32084eb0-f71c-42c6-afa3-7a658e764532-uimage.png}
\begin{figure*}[!htbp]
\centering \includegraphics{Thalamus/images/image.png}
\makeatother 
\caption{Effect of initialization on the final accuracy and convergence curves. Validation accuracy (Dice) and training loss for a network with random initialization vs. initialization from a network trained on a 3T dataset are shown. The segmentation results indicate a statistically significant improvement over two nuclei (VA and Hb) when initialized from the 3T network.}
\label{f-67347f0d82dc}
\end{figure*}
\egroup




\subsection{Network Performance}Figure~\ref{f-1b1c9f8fc2d4} compares shows Dice for (a) WMn-MPRAGE data segmented using the proposed method (blue) and THOMAS (red) and (b) CSFn-MPRAGE data segmented using the proposed method (blue) and FreeSurfer (red). Note that this aggregates Dice over control, MS, and ET cases (WMn: n=40 and CSFn: n=33). Our proposed CNN-based method shows improved Dice over THOMAS and FreeSurfer for seven and ten nuclei, respectively. 
\bgroup
\fixFloatSize{images/69896b15-2e20-4a41-9e30-73c851da7a6e-uslide3.jpeg}
\begin{figure*}[!htbp]
\centering \includegraphics{Thalamus/images/slide3.jpeg}
\makeatother 
\caption{Comparison of average Dice for a) WMn-MPRAGE and b) CSFn-MPRAGE for the proposed method (blue) and THOMAS/FreeSurfer (red), aggregated over all cases. Dice values separated by disease type are shown in Tables~\ref{tw-2899300b0892}, \ref{tw-81cd74d7f41f}, \ref{tw-b701e1e94362} and~\ref{tw-f86873d1e72f}.}
\label{f-1b1c9f8fc2d4}
\end{figure*}
\egroup




\subsubsection{WMn-MPRAGE Data}Tables~\ref{tw-2899300b0892} and~\ref{tw-81cd74d7f41f} report the mean Dice and VSI in the proposed method against THOMAS in WMn-MPRAGE data separately for control subjects (n=13), MS patients (n=15) and ET patients (n=12). Note that the thalamic nuclei are organized in descending order of size, with the smaller nuclei ({\textless}200 mm\ensuremath{^{3}}) shaded gray. This format is used for all subsequent tables. Average Dice for the ET subjects showed significant improvements for the whole thalamus and 5 nuclei compared to THOMAS, ranging from 0.02 (whole thalamus) to 0.17 (VLa). Average VSI showed improvements for four nuclei, ranging from 0.05 (VLp) to 0.18 (VLa). For control and MS subjects, Dice and VSI were largely comparable, with a modest but statistically significant improvement in mean Dice over 5 nuclei (range 0.03-0.06) and 3 nuclei (0.01-0.04) for control and MS subjects, respectively. The average VSI showed improvements for 2 nuclei (0.11) while performing worse for 1 nucleus (0.03) for control subjects and no differences for patients with MS.

The mean Dice and VSI of the proposed method in 3T WMn-MPRAGE data from ET patients (the same patients that were scanned at 7T) did not show statistically significant differences in Dice or VSI between 7T and 3T data, attesting to the ability of the network to be useful in clinically relevant field strengths. 

Figure~\ref{f-4d03d5d032a5}  shows segmentation results from a patient with ET scanned on a 7T (top panel) and a 3T MRI scanner (bottom panel) using WMn-MPRAGE. Representative axial and coronal slices with automated segmentation labels overlaid for both the left and right thalamus are shown on the right. The increased SNR and $B_1$ inhomogeneity in the 7T image (white arrow) can be clearly seen at the periphery, while the inter-nuclear contrast seems comparable. The Dices (and hence the segmentations) are virtually identical as shown in Tables~\ref{tw-2899300b0892} and~\ref{tw-81cd74d7f41f}.
\begin{table*}[!htbp]
\caption{Comparison of mean Dice for THOMAS vs. the proposed method for WMn-MPRAGE data. Median volumes in mm\ensuremath{^{3}} are shown in parentheses and nuclei with {\textless}200 mm\ensuremath{^3} shaded, gray indicates {\textgreater}10\% change).}
\label{tw-2899300b0892}
\def\arraystretch{1}
\ignorespaces 
\centering 
\begin{tabulary}{\linewidth}{LLLLLLLL}
\tbltoprule 
Disease &
  \multicolumn{2}{p{\dimexpr(\mcWidth{2}+\mcWidth{3})}}{Control} &
  \multicolumn{2}{p{\dimexpr(\mcWidth{4}+\mcWidth{5})}}{MS} &
  \multicolumn{2}{p{\dimexpr(\mcWidth{6}+\mcWidth{7})}}{ET-7T } &
  ET-3T \\
Nuclei (Volume) &
  THOMAS (n=13) &
  CNN (n=13) &
  THOMAS (n=15) &
  CNN (n=15) &
  THOMAS (n=12)  &
  CNN (n=12)  &
  CNN (n=12)\\\cline{1-1}\cline{2-2}\cline{3-3}\cline{4-4}\cline{5-5}\cline{6-6}\cline{7-7}\cline{8-8}
Thalamus (5515) &
  0.92 &
  0.92 &
  0.92 &
  0.92 &
  0.88  &
  0.90*  &
  0.90 \\
Pul (1286) &
  0.87 &
  0.87 &
  0.85 &
  0.86* &
  0.84  &
  0.85  &
  0.84 \\
VLp (882) &
  0.79 &
  0.79 &
  0.79 &
  0.78 &
  0.75  &
  0.80*  &
  0.79 \\
MD-Pf (686) &
  0.86 &
  0.85 &
  0.85 &
  0.86 &
  0.83  &
  0.84  &
  0.84 \\
VPl (346) &
  0.68 &
  0.71* &
  0.69 &
  0.69 &
  0.57 &
  \cellcolor[HTML]{E5E5E5}{0.64*} &
  0.63\\
VA (301) &
  0.71 &
  0.74* &
  0.69 &
  0.72* &
  0.58  &
  \cellcolor[HTML]{E5E5E5}{0.66*} &
  0.63 \\
\cellcolor[HTML]{E5E5E5}{AV (146) } &
  0.77  &
  0.79  &
  0.78  &
  0.77  &
  0.64  &
  \cellcolor[HTML]{E5E5E5}{0.71*} &
  0.71 \\
\cellcolor[HTML]{E5E5E5}{VLa (121) } &
  0.65  &
  0.71*  &
  0.64  &
  0.66  &
  0.47  &
  \cellcolor[HTML]{E5E5E5}{0.64* } &
  0.63 \\
\cellcolor[HTML]{E5E5E5}{CM (119) } &
  0.75  &
  0.75  &
  0.74  &
  0.75  &
  0.63  &
  0.66  &
  0.66 \\
\cellcolor[HTML]{E5E5E5}{LGN (115) } &
  0.75  &
  0.76  &
  0.71  &
  0.69  &
  0.55 &
  0.59 &
  0.59\\
\cellcolor[HTML]{E5E5E5}{MGN (76) } &
  0.68  &
  0.74*  &
  0.75  &
  0.74  &
  0.63  &
  0.65  &
  0.65 \\
\cellcolor[HTML]{E5E5E5}{MTT (49) } &
  0.67  &
  0.71  &
  0.69  &
  0.71  &
  0.67  &
  0.66  &
  0.64 \\
\cellcolor[HTML]{E5E5E5}{Hb (29) } &
  0.76  &
  0.80*  &
  0.72  &
  0.76*  &
  0.76  &
  0.74  &
  0.72 \\
\tblbottomrule 
\end{tabulary}\par 
\end{table*}

\begin{table*}[!htbp]
\caption{Comparison of mean VSI for THOMAS vs. proposed method for WMn-MPRAGE data (median volumes in mm\ensuremath{^3} are shown in parentheses and nuclei with {\textless}200 mm\ensuremath{^3} shaded, gray indicates {\textgreater}10\% change)}
\label{tw-81cd74d7f41f}
\def\arraystretch{1}
\ignorespaces 
\centering 
\begin{tabulary}{\linewidth}{LLLLLLLL}
\tbltoprule Disease & \multicolumn{2}{p{\dimexpr(\mcWidth{2}+\mcWidth{3})}}{Control} & \multicolumn{2}{p{\dimexpr(\mcWidth{4}+\mcWidth{5})}}{MS} & \multicolumn{2}{p{\dimexpr(\mcWidth{6}+\mcWidth{7})}}{ET-7T} & ET-3T\\
\tblmidrule 
Nuclei (Volume) &
  THOMAS (n=13) &
  CNN (n=13) &
  THOMAS (n=15) &
  CNN (n=15) &
  THOMAS (n=12)  &
  CNN (n=12)  &
  CNN (n=12)\\
Thalamus (5515) &
  0.97 &
  0.97 &
  0.98 &
  0.98 &
  0.94  &
  0.95  &
  0.96\\
Pul (1286) &
  0.96 &
  0.96 &
  0.95 &
  0.95 &
  0.95  &
  0.95  &
  0.95\\
VLp (882) &
  0.97* &
  0.94 &
  0.93 &
  0.93 &
  0.89  &
  0.94*  &
  0.95\\
MD-Pf (686) &
  0.94 &
  0.92 &
  0.95 &
  0.95 &
  0.93  &
  0.93  &
  0.94\\
VPl (346) &
  0.90 &
  0.92 &
  0.92 &
  0.91 &
  0.86 &
  0.92 &
  0.92\\
VA (301) &
  0.88 &
  0.89 &
  0.91 &
  0.88 &
  0.92  &
  0.95  &
  0.93\\
\cellcolor[HTML]{E5E5E5}{AV (146) } &
  0.89  &
  0.93  &
  0.93  &
  0.88  &
  0.71  &
  \cellcolor[HTML]{E5E5E5}{0.84* } &
  0.92\\
\cellcolor[HTML]{E5E5E5}{VLa (121) } &
  0.81  &
  \cellcolor[HTML]{E5E5E5}{0.92* } &
  0.87  &
  0.86  &
  0.69  &
  \cellcolor[HTML]{E5E5E5}{0.87* } &
  0.92\\
\cellcolor[HTML]{E5E5E5}{CM (119) } &
  0.91  &
  0.94  &
  0.89  &
  0.89  &
  0.92  &
  0.91  &
  0.92\\
\cellcolor[HTML]{E5E5E5}{LGN (115) } &
  0.91  &
  0.92  &
  0.91  &
  0.89  &
  0.82 &
  0.88* &
  0.90\\
\cellcolor[HTML]{E5E5E5}{MGN (76) } &
  0.79  &
  \cellcolor[HTML]{E5E5E5}{0.90* } &
  0.90  &
  0.87  &
  0.77  &
  0.83  &
  0.86\\
\cellcolor[HTML]{E5E5E5}{MTT (49) } &
  0.91  &
  0.88  &
  0.91  &
  0.89  &
  0.89  &
  0.87  &
  0.89\\
\cellcolor[HTML]{E5E5E5}{Hb (29) } &
  0.88  &
  0.92  &
  0.90  &
  0.91  &
  0.89  &
  0.88  &
  0.87\\
\tblbottomrule 
\end{tabulary}\par 
\end{table*}

\bgroup
\fixFloatSize{Thalamus/images/slide4.jpeg}
\begin{figure*}[!htbp]
\centering \includegraphics{Thalamus/images/slide4.jpeg}
\makeatother 
\caption{WMn-MPRAGE images of a patient with ET acquired at 7T (top left) and 3T (bottom left). The zoomed inset images on the right show representative axial and coronal sections with and without automatic segmentation overlays. Automated thalamic nuclei segmentations are almost identical to the manual segmentation shown for the left thalamus in outlines. Note the clear visualization of small structures like the MTT and habenula on both sets of images. The increased heterogeneity of B1 at 7T can be clearly seen (white arrow), but the cropped images are comparable. }
\label{f-4d03d5d032a5}
\end{figure*}
\egroup




\subsubsection{CSFn-MPRAGE Data }Tables~\ref{tw-b701e1e94362} and~\ref{tw-f86873d1e72f} show the average Dice and VSI for the proposed method and FreeSurfer in CSFn-MPRAGE data separately for 6 healthy control subjects, 15 patients with MS and 12 patients with ET. The proposed method significantly outperformed FreeSurfer in all nuclei and disease types with an average improvement in Dice of 0.33, 0.26, and 0.30 for control, MS, and ET subjects, respectively. VSI showed improvements over 3, 6, and 4 nuclei with an average improvement of 0.39, 0.37, and 0.23 over control, MS, and ET subjects, respectively.


\begin{table*}[!htbp]
\caption{Comparison of the mean Dice for FreeSurfer vs. the proposed method for CSFn-MPRAGE data. Median volumes in mm\ensuremath{^{3}} are shown in parentheses.}
\label{tw-b701e1e94362}
\def\arraystretch{1}
\ignorespaces 
\centering 
\begin{tabulary}{\linewidth}{LLLLLLL}
\tbltoprule 
Disease &
  \multicolumn{2}{p{\dimexpr(\mcWidth{2}+\mcWidth{3})}}{Control} &
  \multicolumn{2}{p{\dimexpr(\mcWidth{4}+\mcWidth{5})}}{MS} &
  \multicolumn{2}{p{\dimexpr(\mcWidth{6}+\mcWidth{7})}}{ET}\\
Nuclei (volume) &
  FreeSurfer (n=5)  &
  CNN (n=6)  &
  FreeSurfer (n=14)  &
  CNN (n=15)  &
  FreeSurfer (n=11)  &
  CNN (n=12) \\
Thalamus (5279) &
  0.81 &
  0.90* &
  0.82 &
  0.90* &
  0.86 &
  0.90*\\
Pul (1202) &
  0.69 &
  0.83* &
  0.67 &
  0.83* &
  0.70 &
  0.83*\\
VLp (843) &
  0.54 &
  0.79* &
  0.57 &
  0.78* &
  0.56 &
  0.79*\\
MD-Pf (670) &
  0.56 &
  0.83* &
  0.63 &
  0.84* &
  0.68 &
  0.83*\\
VPl (337) &
  0.45 &
  0.64* &
  0.45 &
  0.66*  &
  0.45 &
  0.65*\\
VA (286) &
  0.40 &
  0.71* &
  0.45 &
  0.71* &
  0.48 &
  0.67*\\
AV (138)  &
  0.37  &
  0.77* &
  0.38  &
  0.71* &
  0.47  &
  0.70*\\
VLa (116)  &
  0.21  &
  0.64* &
  0.20  &
  0.61* &
  0.19  &
  0.65*\\
CM (110)  &
  0.33  &
  0.67* &
  0.42  &
  0.69* &
  0.35  &
  0.65*\\
LGN (108)  &
  0.08  &
  0.64* &
  0.09  &
  0.59*  &
  0.20  &
  0.53*\\
MGN (70)  &
  0.04  &
  0.67* &
  0.06  &
  0.69* &
  0.07  &
  0.65*\\
MTT (49)  &
   &
  0.50  &
   &
  0.58  &
   &
  0.60 \\
Hb (28)  &
   &
  0.67  &
   &
  0.67  &
   &
  0.69 \\
\tblbottomrule 
\end{tabulary}\par 
\end{table*}

\begin{table*}[!htbp]
\caption{Comparison of mean VSI for FreeSurfer vs. proposed method for CSFn-MPRAGE data Median volumes in mm\ensuremath{^{3}} are shown in parentheses.}
\label{tw-f86873d1e72f}
\def\arraystretch{1}
\ignorespaces 
\centering 
\begin{tabulary}{\linewidth}{LLLLLLL}
\tbltoprule 
Disease &
  \multicolumn{2}{p{\dimexpr(\mcWidth{2}+\mcWidth{3})}}{Control} &
  \multicolumn{2}{p{\dimexpr(\mcWidth{4}+\mcWidth{5})}}{MS} &
  \multicolumn{2}{p{\dimexpr(\mcWidth{6}+\mcWidth{7})}}{ET}\\
Nuclei (volume) &
  FreeSurfer (n=5)  &
  CNN (n=6)  &
  FreeSurfer (n=14)  &
  CNN (n=15)  &
  FreeSurfer (n=11)  &
  CNN (n=12) \\
Thalamus (5279) &
  0.90 &
  0.96 &
  0.88 &
  0.97* &
  0.94 &
  0.96\\
Pul (1202) &
  0.93 &
  0.94 &
  0.84 &
  0.93* &
  0.88 &
  0.93\\
VLp (843) &
  0.96 &
  0.94 &
  0.93 &
  0.93 &
  0.92 &
  0.93\\
MD-Pf (670) &
  0.85 &
  0.94 &
  0.79 &
  0.93* &
  0.92 &
  0.95\\
VPl (337) &
  0.58 &
  0.87*  &
  0.57 &
  0.88*  &
  0.58 &
  0.94* \\
VA (286) &
  0.75 &
  0.88 &
  0.81 &
  0.88* &
  0.77 &
  0.92*\\
AV (138)  &
  0.94  &
  0.91  &
  0.86  &
  0.85  &
  0.89  &
  0.90 \\
VLa (116)  &
  0.28  &
  0.89* &
  0.28  &
  0.88* &
  0.31  &
  0.90*\\
CM (110)  &
  0.59  &
  0.87* &
  0.60  &
  0.90* &
  0.49  &
  0.88*\\
LGN (108)  &
  0.81  &
  0.80  &
  0.83  &
  0.78  &
  0.87  &
  0.85 \\
MGN (70)  &
  0.77  &
  0.87  &
  0.77  &
  0.83  &
  0.87  &
  0.88 \\
MTT (49)  &
   &
  0.85  &
   &
  0.85  &
   &
  0.89 \\
Hb (28)  &
   &
  0.87  &
   &
  0.84  &
   &
  0.87 \\
\tblbottomrule 
\end{tabulary}\par 
\end{table*}
Images from a representative MS patient acquired using two different contrasts (WMn-MPRAGE and CSFn-MPRAGE) along with their overlaid segmentations are shown in Figure~\ref{f-7433626f60be} . We can clearly see that the network segmented the thalamic nuclei fairly accurately in the CSFn-MPRAGE image, despite its seemingly poor intrathalamic contrast. For the larger nuclei, the Dice values for these two subjects ranged from 0.69-0.86 and 0.66-0.89 for WMn-MPRAGE and CSFn-MPRAGE, respectively, and for the smaller nuclei, the Dice values ranged from 0.73-0.82 and 0.61-0.81, respectively. Note also that the presence of MS lesions (arrows) has not affected network performance, demonstrating the robustness of the training process that included healthy and MS patients.
\bgroup
\fixFloatSize{Thalamus/images/slide5.jpeg}
\begin{figure*}[!htbp]
\centering \includegraphics{Thalamus/images/slide5.jpeg}
\makeatother 
\caption{WMn-MPRAGE images (top left) and CSFn-MPRAGE images (bottom left) of a patient with MS acquired at 7T. The zoomed inset images on the right show representative axial and coronal sections with and without automated segmentation overlays with manual segmentation shown for the left thalamus in outlines. Note the comparable thalamic nuclei segmentations for both cases despite the poor intrathalamic nuclear contrast in the CSFn-MPRAGE image. MS lesions are shown as white arrows in the top inset image. }
\label{f-7433626f60be}
\end{figure*}
\egroup




\subsection{Network Robustness to Noise}Figure~\ref{f-136457b1c04a}  shows the effect of the addition of noise on the performance of the proposed method. It can be seen that even though there is a gradual decline in Dice for whole thalamus and all nuclei (except for VPL and LGN), there is no significant reduction in accuracy until SNR = 11.4 (i.e. reduced {\texttildeapprox} 2X from the baseline), showing that further acquisition speed-up (which usually comes at the cost of SNR) can be achieved. From an SNR of 11.4 to an SNR of 8 ({\texttildeapprox}3X reduction from baseline), two nuclei (MTT and LGN) and the entire thalamus show a significant decline while the remaining show a more modest decline.
\bgroup
\fixFloatSize{Thalamus/images/slide6.jpeg}
\begin{figure*}[!htbp]
\centering \includegraphics{Thalamus/images/slide6.jpeg}
\makeatother 
\caption{Effect of noise addition on the performance of the proposed method. Note the excellent performance in most nuclei for a 2X lower SNR (SNR = 11.4) compared to baseline (SNR = 23).}
\label{f-136457b1c04a}
\end{figure*}
\egroup


\subsection{Clinical Analysis in Multiple Sclerosis}Two-tailed t-test analyses showed no statistically significant difference between the automatically segmented nuclei of the left and right thalami, except for MTT, and thus the average volume of the left and right thalami was used for the analysis. An ANCOVA was conducted to compare the absolute bilateral thalamic nuclear volumes between the two groups, controlling for age, gender, ICV and PI. There was significant atrophy in MS patients in the entire thalamus (F\unskip~\cite{1643371:26789934,1643371:26789933} = 9.5, p = 0.005), AV (F\unskip~\cite{1643371:26789934,1643371:26789933} = 26.6, p \textless\ 0.0001), Pul (F\unskip~\cite{1643371:26789934,1643371:26789933} = 18, p = 0.0003) and MGN (F\unskip~\cite{1643371:26789934,1643371:26789933} = 17, p = 0.0004). These were identical to the results observed using THOMAS segmented data, except for an additional atrophy in Hb (F\unskip~\cite{1643371:26789934,1643371:26789933} = 9.1, p = 0.006) which did not survive significance after Bonferroni correction.
    
\section{DISCUSSION}
Many studies have explored the use of CNNs in subcortical segmentation \unskip~\cite{1643371:26789955,1643371:26789913,1643371:26789947,1643371:26789902} but we believe this is the first work to use CNNs to segment the thalamic nuclei from structural MRI data. A single network is introduced for the segmentation of WMn-MPRAGE data for 3T and 7T, as well as for healthy controls and patients with MS and ET. Twelve thalamic nuclei including small structures such as the MTT, habenula, and lateral and medial geniculate nuclei were segmented in less than a minute. A fine-tuning approach was employed to incorporate information learned from the WMn network to segment thalamic nuclei from CSFn-MPRAGE images, which have poor intra-thalamic contrast. Although Dice values were slightly lower than those of the corresponding WMn-MPRAGE images acquired in the same patients, they were significantly improved compared to FreeSurfer with significantly reduced processing times (1 minute vs. several hours). The proposed method also exhibited robustness to noise (up to 2X lower SNR compared to baseline). 

Previous neuroimaging studies, although limited to the whole thalamus, have shown evidence of thalamic involvement in MS. Planche \unskip~\cite{1643371:26789959} recently demonstrated atrophy of specific thalamic nuclei due to MS. Our method showed statistically significant atrophy in MS patients compared to healthy subjects for the entire thalamus, as well as for AV, MGN and pulvinar nuclei, which is well in agreement with the results of Planche \unskip~\cite{1643371:26789959}. The anteroventral nucleus is a critical component in episodic memory and the circuit of Papez \unskip~\cite{1643371:26789934}. Our network successfully segmented this nucleus with a mean Dice of {\textgreater} 0.71 for both WMn-MPRAGE and CSFn-MPRAGE. The latter is of critical relevance for analyzing public databases such as the Alzheimer's disease neuroimaging initiative (ADNI) (which only has conventional CSFn-MPRAGE data), to study the effect of Alzheimer's disease on nuclei such as AV and MD, that are critically involved in episodic memory. Fast and accurate segmentation of the VIM nucleus using the proposed method will allow integration of our work with clinical applications such as deep brain stimulation surgery \unskip~\cite{1643371:26789937,1643371:26789916,1643371:26789943} and high intensity focused ultrasound treatment of essential tremor\unskip~\cite{1643371:26789906}, which targets the VIM nucleus. The use of conventional MPRAGE in neuroimaging protocols, as well as the recent availability of WMn-MPRAGE on major vendor platforms, should allow a faster clinical adoption of our technique. We have optimized the 3T WMn-MPRAGE pulse sequence to perform comparably to 7T, with some time penalty \unskip~\cite{1643371:26789912}. While we have shown a comparable segmentation performance between 7T and 3T WMn-MPRAGE data in a small cohort of patients in this study, more validation studies would be needed in 3T with different disease populations to establish the sufficiency of 3T imaging. A larger study is currently underway to test the ability of our method to detect atrophy in patients with AUD and dementia, especially using conventional 3T MPRAGE. 

The recent shape-based segmentation method of Liu\unskip~\cite{1643371:26789966} is one of the few methods that segments MPRAGE data using the Morel atlas convention. Although their method generated 23 nuclei, the lateral and medial geniculate nuclei were notably absent from their output. During manual segmentation, we chose to merge some of the smaller nuclei with larger contiguous nuclei, resulting in a smaller number of reported nuclei compared to their method (e.g. MD is combined with Pf, and the pulvinar complex is segmented as a single nucleus in our manual segmentation). Our method's performance on the WMn-MPRAGE data shows comparable Dice values for most nuclei, lower Dice for VPL and VA nuclei, and higher Dice for the CM nucleus and MTT, compared to Liu's shape-based method. However, our network's performance on CSFn-MPRAGE data shows lower accuracy in comparison, presumably because of the lack of shape information, which could help in the face of poor intra-thalamic contrast. Our analysis had a much larger spread of cases that included healthy controls (n = 13), as well as patients with ET (n=12) and MS (n=15), compared to 9 healthy subjects in Liu\unskip~\cite{1643371:26789966} . The proposed method shows a dramatically improved Dice accuracy over the Bayesian atlas-based method of Iglesias\unskip~\cite{1643371:26789917} for all nuclei and disease types while significantly shortening processing times to less than a minute from several hours.

The proposed CNN-based method has limitations that are common to most deep learning methods. Data diversity during the training phase is critical in creating a generalizable network. The current network was trained using images acquired from 3T and 7T and patients with MS and ET in addition to healthy subjects. However, since the network has not been exposed to images with metal artifacts from surgical clips or deep brain stimulation electrodes, it is likely to fail under those conditions and will require special training. The performance of this network in other diseases such as Alzheimer's disease needs to be evaluated. Limitations specific to our implementation include not taking advantage of 3D data; with sufficient 3D training data and memory, a 3D cascaded network will likely improve the accuracy of our method by fully leveraging the 3D structural information. Due to the lack of manual segmentation data, the performance of the proposed method on 3T CSFn-MPRAGE data was not investigated. We also observed a slight reduction in performance in the CSFn-MPRAGE data set compared to that of the WMn-MPRAGE data set for the smaller nuclei. This could be due in part to the inherently lower intra-thalamic contrast in conventional CSFn-MPRAGE images. Future work will explore the synthesizing of WMn-MPRAGE images from CSFn-MPRAGE using contrast synthesis methods\ensuremath{^{}}and then applying the WMn-MPRAGE optimized network for better accuracy. 
    
\section{CONCLUSION}
We have proposed the use of a CNN-based cascaded multiclass multiplanar method for the segmentation of thalamic nuclei and evaluated it on images with different contrasts and magnetic field strengths for both healthy and diseased populations. This method has been applied successfully to both advanced MR acquisition techniques with high intra-thalamic contrast (WMn-MPRAGE) and the more commonly used low intra-thalamic contrast sequences (CSFn-MPRAGE). Furthermore, the effectiveness of this method in real-life applications has been investigated through a clinical analysis of volume atrophy in MS patients. 
    
\section{ABBREVIATIONS}
AV Anteroventral nucleus

CM Centromedian nucleus 

DTI Diffusion Tensor Imaging 

Hb Habenular nucleus 

LGN Lateral Geniculate nucleus 

MD Mediodorsal nucleus 

MGN Medial Geniculate nucleus 

MTT Mammillothalamic Tract 

Pul Pulvinar nucleus 

VA Ventral Anterior nucleus 

VIM Ventralis Intermedius nucleus 

VLa Ventral Lateral Anterior nucleus 

VLp Ventral Lateral Posterior nucleus 

VPL Ventral Posterolateral nucleus 

WM White Matter 

WMn White Matter Nulled

CSF Cerebrospinal Fluid 

CSFn Cerebrospinal Fluid Nulled 

MPRAGE Magnetization-Prepared Rapid Gradient Echo 


\section{ACKNOWLEDGEMENTS} We would like to acknowledge the funding support from the National Institutes of Health (R21 AA023582-01) and the Arizona Alzheimer's Consortium. 


% \section*{APPENDIX}

\bibliographystyle{ieeetr}
\bibliography{Chapters/Thalamus/thalamus}

\end{document}
