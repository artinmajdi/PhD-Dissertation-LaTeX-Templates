% \renewcommand{\chapterpath}[1]{Chapters/drive_net/#1}
\renewcommand{\figurepath}[1]{Chapters/drive_net/figures/#1}

\chapter{The Principles and Application of Transfer Learning}\label{ch:transfer_learning}

Transfer Learning constitutes a technique within the realm of machine learning, where a model devised for one task is repurposed as a foundation for a different task. It has gained considerable popularity within deep learning contexts, primarily due to its ability to proficiently train deep neural networks with relatively limited data. This characteristic is profoundly beneficial, considering that most real-world applications do not possess the luxury of millions of annotated data points necessary for training intricate models.

Transfer learning can be implemented via two primary strategies within deep learning contexts:

\begin{itemize}
    \item \textbf{Pre-training and fine-tuning:} This strategy employs a pre-trained model, such as ResNet or VGG, which has been trained on extensive datasets like ImageNet, and uses it as a foundation for subsequent fine-tuning, tailored to the specific task at hand.
    \item \textbf{Feature extraction:} This alternative strategy involves using a pre-trained model as a sophisticated feature extractor. The output of the pre-trained model serves as the input for training another distinct machine learning model.
\end{itemize}

Transfer learning provides several notable advantages:

\begin{itemize}
    \item \textbf{Efficiency:} By negating the need for training a model from scratch, transfer learning significantly reduces computational time and resource demands.
    \item \textbf{Lower requirement of training data:} When the available training data for the task of interest is scarce, transfer learning proves particularly beneficial.
    \item \textbf{Improved performance:} Provided there is a correlation between the pre-training and fine-tuning tasks, transfer learning can enhance the model's performance on the target task.
\end{itemize}

However, there exist certain pitfalls to the application of transfer learning:

\begin{itemize}
    \item \textbf{Negative transfer:} This phenomenon transpires when the knowledge transferred from the source task negatively impacts the performance on the target task, typically when the source and target tasks are unrelated.
    \item \textbf{Domain discrepancy:} If the source and target domains exhibit substantial differences, transfer learning may not yield satisfactory results.
\end{itemize}

As for the hybrid approach combining Convolutional Neural Networks (CNNs) and conventional classifiers such as Random Forests, the rationale is to utilize the strengths of both models. CNNs exhibit powerful feature extraction capabilities, particularly with image data, by capturing spatial relationships and hierarchically extracting high-level features. However, they can sometimes be perceived as black boxes due to their multitude of parameters.

Conversely, Random Forests, as an ensemble learning method, are highly robust against overfitting and offer superior interpretability. This combination facilitates the extraction of complex features via CNNs and the subsequent use of these features in a model that is both robust and interpretable via Random Forests. Furthermore, the ability of Random Forests to handle tabular data, which can prove challenging for CNNs, renders the hybrid model more versatile.

%% ############################################# %%
%% A P P L I C A T I O N 1 :  D R I V E - N E T  %%
%% ############################################# %%
\section{Application 1: Drive-Net: Convolutional Network for Driver Distraction Detection}\label{ch:drive-net}

\section{Abstract}
To help prevent motor vehicle accidents, there has been significant interest in finding an automated method to recognize signs of driver distraction, such as talking to passengers, fixing hair and makeup, eating and drinking, and using a cell phone. In this paper, we present an automated supervised learning method called Drive-Net for driver distraction detection. Drive-Net uses a combination of a convolutional neural network (CNN) and a random decision forest to classify images of a driver. We compare the performance of our proposed Drive-Net to two other popular machine learning approaches: a recurrent neural network (RNN) and a multilayer perceptron (MLP). We tested the methods on a publicly available database of images acquired in a controlled environment that contained about 22425 images manually annotated by an expert. The results show that Drive-Net achieves a detection accuracy of 95\%, which is 2\% more than the best results obtained in the same database using other methods.


\textbf{KEYWORDS:\ } Image classification, convolutional neural networks, random forest, driver distraction
%

\newpage
\section{Introduction}
Distracted driving is a major cause of motor vehicle accidents. Each day in the United States, approximately 9 people are killed, and more than 1000 are injured in crashes that involve a distracted driver~\cite{schroeder_National_2018}. It is estimated that about 25\% of fatalities in motor vehicle accidents are due to distracted driving~\cite{website_insurer_2013}. A study of American motor vehicle fatalities~\cite{schroeder_National_2018} reveals the top 10 causes of distracted driving:

\begin{enumerate}
\item Generally distracted or ``lost in thought'' -{\textendash} 62\%
\item Cell phone use {\textemdash} 12\%
\item Outside person, object, or event {\textendash} 7\%.
\item Other occupants -{\textendash} 5\%.
\item Using or reaching for a device brought into the car (e.g., phone) {\textendash} 2\%.
\item Eating or drinking -{\textendash} 2\%.
\item Adjusting audio or climate controls {\textemdash} 2\%.
\item Using devices to operate the vehicle (e.g., adjusting mirrors or seatbelts) {\textemdash} 1\%.
\item Moving objects (e.g., insects or pets) -{\textendash} 1\%
\item Smoking related {\textemdash} 1\%.
\end{enumerate}

Therefore, there is interest in using dashboard camera image analysis to automatically detect drivers engaged in distracting behavior. A dataset of such dashboard camera images, observing various activities of drivers, has been compiled and used for the Kaggle competition regarding automated detection of driver distraction~\cite{montoya_State_2016}. Figure~\ref{fig:drive-net/figure1} shows examples of some images from the dashboard camera manually annotated as different activities while driving.

\begin{figure}[!htbp]
    \centering
    \includegraphics[width=\textwidth]{\figurepath{image1.jpeg}}%
    \caption[Examples of Driver Distractions Captured by Dashboard Camera]{Representative dashboard camera images of drivers being distracted. From left to right: drinking while driving; safe driving; reaching behind while driving; talking on the phone (left hand) while driving~\cite{montoya_State_2016}.}%
    \label{fig:drive-net/figure1}
\end{figure}

Object detection and human behavior detection are well-researched topics in the computer vision literature~\cite{borji_Salient_2019}. Machine learning (esp. Deep Learning) techniques can often learn complex models and achieve high accuracy, so many researchers have started to apply such techniques to solve computer vision problems, including object detection and human behavior detection. For example, the Inception-v4 model proposed by Szegedy~\cite{szegedy_InceptionV4_2017} is a supervised learning model made up of deep convolutional residual networks (ResNet) that have more than 75 trainable layers, and achieves 96.92\% accuracy in the ImageNet data set. Girshick~\cite{girshick_RegionBased_2016} introduced a very powerful method for object detection and segmentation using a region-based convolutional neural network (CNN). This method divides the human behavior detection problem into two problems. First, they apply an object detection algorithm to detect the regions of interest (ROIs) where people are present within an image. Next, each ROI is fed to CNN to identify the type of behavior exhibited in the given ROI\@. Adding other traditional machine learning methods, such as ensemble learning (i.e.\ bagging) and K nearest neighbors (KNN), to the CNN model is a way to improve the accuracy of the already existing model~\cite{kim_Vehicle_2017}.

\begin{figure*}[!htbp]
    \centering
    \includegraphics[width=\textwidth]{\figurepath{image2.jpeg}}%
    \caption[Overview of the Proposed Drive-Net: A CNN Architecture Coupled with a Random Forest Classifier]{An overview of the proposed Drive-Net. Our proposed CNN architecture (shown on the left side) consists of two convolution layers (conv), each followed by a maxpooling layer (pool), and a final ReLU layer, the output of which is regularized using dropouts to obtain a fully connected layer (FC). The FC layer is fed as input to the random forest classifier (on the right side), which predicts the final class label.}%
    \label{Drive-Net/figure2}
\end{figure*}

One of the main drawbacks of CNN is that training the network using a large data set can lead to overfitting the model. To avoid this, ensemble methods such as random decision forests can be effective. With this in mind, we propose a new supervised learning algorithm called Drive-Net that combines a CNN and a random forest in a cascading fashion for application to the problem of driver distraction detection using dashboard camera images. We compare our proposed Drive-Net to two other neural network methods: a residual neural network (RNN) and a multilayer perceptron (MLP). We show that Drive-Net achieves better classification accuracy than the driver distraction detection algorithms that were proposed in the Kaggle competition~\cite{montoya_State_2016}.

\section{Methods}
Our proposed method, Drive-Net, is a cascaded classifier consisting of two stages: a CNN as the first stage, whose output layer is fed as input to a random decision forest to predict the final class label. We define each stage in detail below.


\subsection{Convolutional Neural Network Configurations}
We adopted the U-Net architecture~\cite{ronneberger_UNet_2015} as the basis for our CNN\@. The motivation behind this architecture is that the contracting path captures the context around the objects to provide a better representation of the object compared to architectures such as AlexNet~\cite{krizhevsky_One_2014} and VGGNet~\cite{simonyan_Very_2014} Very large networks like AlexNet and VGGNet require learning a massive number of parameters and are very difficult to train in general, needing significant computational time. Thus, we empirically modified the U-Net architecture in this work to suit our application.

To construct our CNN, we discard U-Net's layers of up-convolution and the last two layers of down-sampling and replace them with a $1\times1 $ convolution instead to obtain a fully connected layer. We use the rectifier activation function~\cite{simonyan_Very_2014} for our CNN as the constant gradient of rectified linear units (ReLU) results in faster learning and reduces the problem of vanishing gradient compared to the hyperbolic tangent (${\tanh{(\bullet)}} $). We implement a maximum-pooling layer instead of average-pooling in the subsampling layer. Furthermore, we observed that performance is better when a ReLU layer was configured with the maximum pool layer, resulting in higher classification accuracy after 50 epochs. We use the $1\times1 $ convolutional filter for the Adam~\cite{kingma_Adam_2014} optimizer. All other parameters, such as the number of layers, the size of the convolutional kernel, the training algorithm, and the number of neurons in the final dense layer, were experimentally determined for our application.

To keep the training time small, we reduced the size of the dashboard camera images by a factor of 10 by making them 64 $\times$ 48 in size and feed them as input to our CNN\@. We do not zero-pad the image patches, as the ROI of human activity is located toward the center of the image. Two consecutive convolutional layers are used in the network. The first convolutional layer consists of 32 kernels of size $5\times5\times1 $. The second convolutional layer consists of 64 kernels of size $5\times5\times32 $. The subsampling layer is set as the maximum value in non-overlapping windows of size $2\times2 $ (stride of 2). This reduces the size of the output of each convolutional layer by half. After the two convolutional and subsampling layers, we use a ReLU layer, where the activation y for a given input x is obtained as follows.
\begin{equation}
\label{eq:disp-formula-group-e98e45e883854b6aa2d919dcc0573fee}
y=f(x)=\max(0,x)
\end{equation}

A graphical representation of the architecture of the proposed CNN model is shown in Figure~\ref{Drive-Net/figure2} (see the left side).

\subsection{Random Decision Forest}
A random forest classifier consists of a collection of decision tree classifiers combined to predict the class label, where each tree is grown in a randomized fashion. Each decision tree classifier consists of decision (or split) nodes and prediction (or leaf) nodes. The prediction nodes of each tree in the random forest classifier are labeled by the posterior distribution over the image classes~\cite{bosch_Image_2007}. Each decision node contains a test that best splits the space of the data to be classified. An image is classified by sending it down the decision tree and aggregating the posterior distributions that are reached. Most of the time, randomness is added to training at two points: when subsampling the training data and when choosing node tests. Each tree within the random forest classifier is binary and grows top-down. At each node, we select the binary test to maximize the information gain obtained by partitioning the training set $Q$ of image patches into two sets $Q_i$ according to the test.


\begin{equation}
\label{eq:disp-formula-group-d27d73835da3454a817782924579e245}
\Delta E = - \sum_{i} \frac {\vert Q_i \vert} {\vert Q \vert} E (Q_i)
\end{equation}


Here $E(\cdot) $ is the entropy of the set and $\left\vert \cdot\right\vert $ is the size of the set. We repeat this selection process for each decision node until it reaches a certain depth. Many implementations of random forests~\cite{lepetit_Keypoint_2006,winn_Object_2006} use simple pixel-level tests at nodes because it results in faster tree convergence. As we are interested in features that encode shape and appearance, we are interested in spatial correspondence between pixels. Therefore, we use a simple test proposed by Bosch~\cite{bosch_Image_2007} {\textemdash} as a linear classifier on the characteristic vector {\textemdash} at each decision node.

Suppose that $T $ is the set of all trees, $C $ is the set of all classes, and $L $ is the set of all leaves for a given tree. During training, posterior probabilities $P_{t,l}\left(Y\left(I\right)=c\:\right) $ for each class $c\in C $ are found at each leaf node $l\in L $ for each tree $t\in T $. These probabilities are calculated as the ratio of the number of images $I $ of class $c $ that reach a leaf node $l $ to the total number of images that reach that leaf node $l $. $Y\left(I\right) $ is the class label of image $I $. During test time, we pass a new image through every decision tree until it reaches a prediction (or leaf) node, average all the posterior probabilities, and classify the image as


\begin{equation}
\label{eq:disp-formula-group-032e1395024b44bca0feeb46474c0adc}
\widehat Y(I) = \underset c {\arg \max} \left\{\frac{1}{\left\vert T\right\vert}\sum_{t=1}^{\left\vert T\right\vert} P_{t,l} (Y(I) = c) \right\}
\end{equation}
where $l $ is the leaf node reached by the image $I $ in the tree $t $. A graphical representation of the proposed random forest classifier is shown in Figure~\ref{Drive-Net/figure2}  (see the right side).

\section{Experiments and Results}
\subsection{Data set}
The Kaggle competition~\cite{montoya_State_2016} for driver distraction has provided 22425 images for training and 79727 for tests. Since we did not have access to the test labels, our experiments were carried out solely on the training images. However, the quality and conditions of the training and testing images are similar; the only difference is that none of the drivers used in the training data set appear in the images in the test data set. The images are of size 640 \ensuremath{\times} 480, and for our experiments, we converted them from color to grayscale.

Ten classes are provided, related to the ones listed in Section I\@. Each class includes almost tens of the data, so that we have a uniform distribution of sample data.

%\begin{multicols}{2}
%\centering
\begin{itemize}
\item c0: safe driving
\item c1: texting (right hand)
\item c2: talking on the phone (right hand)
\item c3: texting (left hand)
\item c4: talking on the phone (left hand)
\item c5: operating the radio
\item c6: drinking
\item c7: reaching behind
\item c8: hair and makeup
\item c9: talking to a passenger
\end{itemize}
%\end{multicols}


\subsection{Algorithm Parameters}
The convolutional neural random forest classifier is implemented using TensorFlow~\cite{abadi_tensorflowlarge_2016} and runs on an NVIDIA GeForce GTX TITAN X GPU with 16 GB of memory. The classifier was trained using the Adam stochastic gradient descent algorithm~\cite{kingma_Adam_2014} to efficiently optimize CNN weights. The weights were normalized using initialization as proposed in Kingma~\cite{kingma_Adam_2014} and updated in a mini-batch scheme of 128 candidates. The biases were initialized with zero, and the learning rate was set at $\alpha = 0.001 $. The exponential decay rates for the estimates for the first and second moments were set as $\beta_1 = 0.9 $ and $\beta_2 = 0.99 $, respectively. We used $\epsilon = {10}^{-8} $ to prevent division by zero. A dropout rate of $0.5 $ was implemented as regularization, applied to the output of the last convolutional layer and the dense layer to avoid overfitting. Finally, we used an epoch size of 50.  SoftMax loss (cross-entropy error loss) was used to measure the error loss. We used 100 estimators and a keep rate of $\gamma = {10}^{-4} $ for the random forest algorithm.


\subsection{Performance Evaluation} We tested the algorithm performance by performing k-fold cross-validation on the entire dataset. For our experiments, we varied the values of $k\in\lbrack2, 5 \rbrack $ and found that the results were consistent enough to indicate that the network is not overfitting. Therefore, we chose $k = 5 $. First, we randomly choose the order of the driver images within the data set. For each fold of the k-fold cross-validation, we chose 80\% of the 22425 images as the training data set and tested the trained model on the remaining 20\% of the images. We ensured that the images from the entire dataset appeared in the test dataset only once in all k-folds, thereby allowing each image to be classified as a test image exactly once.

We compared our proposed Drive-Net with two other neural network classifiers: an RNN classifier~\cite{liang_Recurrent_2015} and an MLP classifier~\cite{haykin_Neural_2009}. We report the classification accuracy, which is defined as the percentage of correct predictions and the number of false positives (a.k.a.\ false detections) for each class, as the figures of merit for comparing the algorithms. For classification precision, we present the results of seven other methods based on support vector machines (SVMs), dimensionality reduction techniques such as principal component analysis (PCA), feature extraction techniques such as histogram of oriented gradients (HOG), very deep convolutional nets such as VGG-16, VGG-GAP, and an ensemble of these two as reported by Zhang~\cite{zhang_Apply_2016} using the same Kaggle dataset of 22425 images.

Table~\ref{tab:drive-net.table.1} shows the mean classification accuracy of the different classifiers as reported by Zhang~\cite{zhang_Apply_2016} and that of the three neural network classifiers that we implemented. From Table~\ref{tab:drive-net.table.1}, we observe that Drive-Net achieves a classification accuracy of 4.8\% points higher than the VGG-16 classifier, $3.7\% $ points higher than a VGG-GAP classifier, 2.4\% points higher than an ensemble of VGG-16 and VGG-GAP classifiers, 3.3\% points higher than the RNN classifier and 13\% points higher than the MLP classifier.

\begin{table}[]
\centering
\caption{Mean Classification Accuracy of the Automated Methods}%
\label{tab:drive-net.table.1}
\begin{tabular}{@{}lr@{}}
\toprule
\textbf{Method}             & \textbf{Accuracy} \\ \midrule
Pixel SVC                   & 18.3\%            \\
SVC + HOG                   & 28.2\%            \\
SVC + PCA                   & 34.8\%            \\
SVC + BBox + PCA            & 40.7\%            \\
VGG-16                      & 90.2\%            \\
VGG-GAP                     & 91.3\%            \\
Ensemble VGG-16 and VGG-GAP & 92.6\%            \\
\multicolumn{2}{l}{\cellcolor[HTML]{C0C0C0}}    \\
MLP                         & 82\%              \\
RNN                         & 91.7\%            \\
Drive-Net                   & 95\%              \\ \bottomrule
\end{tabular}
\end{table}

Table~\ref{tab:drive-net.table.2} shows the number of false classifications for our Drive-Net and for RNN and MLP\@. From Table~\ref{tab:drive-net.table.2}, we observe that our Drive-Net can identify the classes c6 (drinking) and c3 (texting with left hand) with minimum false detections, while the RNN and MLP classifiers have a hard time distinguishing these classes with many false detections, usually higher than the number of false detections in the other classes of these methods. Also, the total number of false detections for our Drive-Net is an order of magnitude smaller than that of the MLP classifier and slightly smaller compared to the RNN classifier.

\begin{table}[]
\centering
\caption{Error Count of Each Class for the Neural Network Methods}%
\label{tab:drive-net.table.2}
\begin{tabular}{@{}lccc@{}}
\toprule
\rowcolor[HTML]{FFFFFF}
Class       & Drive-Net & RNN & MLP  \\ \midrule
c0          & 35        & 48  & 356  \\
c1          & 17        & 34  & 199  \\
c2          & 14        & 31  & 158  \\
c3          & 09        & 47  & 116  \\
c4          & 34        & 30  & 252  \\
c5          & 15        & 14  & 108  \\
c6          & 08        & 18  & 263  \\
c7          & 21        & 10  & 117  \\
c8          & 29        & 46  & 181  \\
c9          & 26        & 46  & 268  \\
\rowcolor[HTML]{EFEFEF}
All Classes & 208       & 324 & 2018 \\ \bottomrule
\end{tabular}
\end{table}

\section{Conclusion}
Distracted driving is one of the main causes of motor vehicle accidents. Therefore, there is a significant interest in finding automated methods to recognize signs of driver distraction from dashboard camera images installed in vehicles. We propose a solution to this problem using a supervised learning framework. Our method, named Drive-Net, combines a CNN and a random forest classifier to recognize the various categories of driver distraction in images. We apply our Drive-Net to a publicly available dataset of images used in a Kaggle competition and show that our Drive-Net achieves better accuracy than the driver-distraction algorithms reported in the competition. We also compared Drive-Net to two other neural network algorithms: an RNN and an MLP algorithm, using the same data set. The results show that DriveNet achieves better detection accuracy compared to the other two algorithms.



%% #################################### %%
%% A P P L I C A T I O N 2 :  C I L I A
%% #################################### %%
\section{Classification of Primary Cilia in Microscopy Images Using Convolutional Neural Random Forests}\label{ch:cilia}

\section{Abstract}
Accurate detection and classification of primary cilia in microscopy images is an essential and fundamental task for many biological studies including diagnosis of primary ciliary dyskinesia. Manual detection and classification of individual primary cilia by visual inspection is time consuming, and prone to induce subjective bias. However, automation of this process is challenging as well, due to clutter, bleed-through, imaging noise, and the similar characteristics of the non-cilia candidates present within the image. We propose a convolutional neural random forest classifier that combines a convolutional neural network with random decision forests to classify the primary cilia in fluorescence microscopy images. We compare the performance of the proposed classifier with that of an unsupervised \emph{k}-means classifier and a supervised multi-layer perceptron classifier on real data consisting of 8 representative cilia images, containing more than 2300 primary cilia using precision/recall rates, ROC curves, AUC, and F$_{\beta}$-score for classification accuracy. Results show that our proposed classifier achieves better classification accuracy.


\textbf{KEYWORDS:\ } Image classification, convolutional neural network, random forests, primary cilia, confocal microscopy.
%

\newpage
\section{Introduction}

Primary cilia are curvilinear non-motile sensory organelles protruding from the surface of many eukaryotic cells that are involved in many cell development and physiological processes. Recent research~\cite{miyoshi_Lithium_2009} has shown that primary cilia length in mammalian cells may change due to extracellular environment stimuli, such as renal injury~\cite{verghese_Renal_2009} or external signaling modules. It has also been reported that renal primary cilia are involved in modulation of the mechanistic target of rapamycin (mTOR) pathway. Furthermore, it has been demonstrated that lithium treatment activates the mTOR pathway in renal collecting duct cells expressing aquaporin 2 (AQP2)~\cite{gao_Rapamycin_2013}. The collecting duct cells in the kidney express primary cilia, as shown in Fig.~\ref{Cilia.Fig.1} \(a\). There is a significant interest developing an automated classifier to accurately and rapidly distinguish between primary cilia located in the AQP2 expressed region and primary cilia located elsewhere.

Manual detection and classification of primary cilia within tissues usually involves a large amount of time, especially for large-scale data processing, and is prone to multiple errors caused by background clutter, non-uniform illumination, imaging noise, and subjective bias. This serves as a motivation to develop an automatic algorithm capable of classifying the primary cilia of interest within the fluorescently labeled microscopy images. In order to classify the primary cilia, we need to first find all the cilia locations within the microscopy images. There are a number of curvilinear-structure detection methods such as the top-hat transform-based detector, ridge detector, steerable detector and the multi-scale variance stabilizing transform (MS-VST) detector that are capable of detecting all the cilia locations within the microscopy image~\cite{ram_dissertation_2017, ram_Vehicle_2016, ram_Three_2018}. Nevertheless, in this paper we analyze the already detected primary cilia, and we focus on classifying them as belonging to an AQP2 expressed region or elsewhere in the microscopy images.
%----------------------------------------------- FIGURE 1------------------------------------------------------%
\begin{figure}[!htbp]
	\centering
    \includegraphics[width= 0.7\textwidth]{\figurepath{fig1}}
	\caption{Microscopy image showing primary cilia which are in red color. (a) Cilia (red color) near AQP2-expressing renal collecting duct (white color). (b) Primary cilium within an AQP2 expressed region that we are interested in. (c) Primary cilium elsewhere in the image.}%
	\vspace{-6mm}%
	\label{Cilia.Fig.1}
\end{figure}
%-------------------------------------------------------------------------------------------------------------------%

Recently, the availability of large amounts of data and significant computational power have rapidly increased the popularity of machine learning (esp.\ deep learning) approaches. Convolutional neural networks (CNNs)~\cite{lecun_deeplearning_2015} have outperformed the state-of-the-art in many computer vision applications~\cite{krizhevsky_Imagenet_2017}. Similarly, the applicability of CNNs has also been investigated for medical image analysis~\cite{gupta_Convolutional_2017}. In particular, their capability to learn discriminative features when trained in a supervised fashion makes them useful for automated detection of structures in, e.g., microscopy images~\cite{gupta_Convolutional_2017}.
%----------------------------------------------- FIGURE 2------------------------------------------------------%
\begin{figure*}[!htbp]
	\centering
    \includegraphics[width=\textwidth]{\figurepath{fig2n.png}}
	\caption{An overview of the proposed convolutional neural random forest classifier. Our proposed CNN model where the feature mapping happens (shown on the left side) consists of two convolutional layers each followed by a maxpooling layer and a final ReLU activation layer, following which a dropout regularization is used to obtain the fully connected layer. The learned features are fed to our random forests classifier (shown on the right side), which have trees with decision nodes (d) (in red color) and leaf nodes (\emph{l}) (in green color). At each leaf node we compute posterior probabilities belonging to each class.}
	\vspace{-4mm}%
	\label{Cilia.Fig.2}
\end{figure*}
%-------------------------------------------------------------------------------------------------------------------%

We propose a convolutional neural random forest classifier -- a novel approach that unifies the appealing representation-learning properties of CNNs with the divide-and-conquer principle of decision trees to classify the primary cilia of interest in fluorescence microscopy images of mouse kidney tissues labeled with AQP2 fluorescent antibodies. Our method differs from the conventional CNNs because we use a random decision forest to provide the final predictions. Our method also differs from traditional random forests as the inputs to these are the features from the CNN, which helps in reducing the uncertainty of the routing decisions of a sample taken at the split nodes of the decision trees. We describe our method in detail and present quantitative results comparing it to an unsupervised \emph{k}-means classifier and a supervised multi-layer perceptron (MLP) classifier for cilia classification.

\section{Methods}

\subsection{Preprocessing}

For classification of primary cilia of interest within our images, we first need to identify all potential primary cilia. In this work, we use the MS-VST algorithm~\cite{bozhang_Wavelets_2008} to extract all potential locations of the primary cilia in the microscopy images. Once we obtain all the candidate primary cilia, for each primary cilium, we extract gray scale image patches of $32 \times 32$ pixels centered at the candidate primary cilium centroid. Such a patch size is chosen to contain the primary cilia ($\sim20-25$ pixels in length), and some background around the cilium ($\sim7$ pixels) to include the context information. These image patches are then fed to a convolutional neural random forest classifier in order to classify whether the primary cilia are located within the AQP2 expressed region or elsewhere within the microscopy images.

\subsection{Convolutional Neural Random Forest}

The convolutional neural random forest classifier consists of two stages: a CNN in the first stage whose output is cascaded and fed in as the input to a random decision forest to predict the final class label. We define in detail each stage below.

\subsubsection{\textbf{CNN Configuration}}

We adopt the U-Net architecture~\cite{ronneberger_UNet_2015} as the basis of our CNN\@. The motivation behind this architecture is that the contracting path captures the context around the objects in order to provide a better representation of the object as compared to architectures such as VGGnet~\cite{simonyan_Very_2014}. Networks like VGGnet are very large networks that require learning a massive number of parameters and are very hard to train in general, needing significant computational time. Thus, we empirically modify the U-Net architecture to suit our application.

To construct our CNN, we discard U-Net's layers of up-convolution and the last two layers of down-sampling and replace them with a $1 \times 1$ convolution instead to obtain a fully connected layer. We use the rectified linear units (ReLU)~\cite{ronneberger_UNet_2015} as the activation function for our CNN as the constant gradient of ReLUs results in faster learning and reduces the problem of vanishing gradient compared to hyperbolic tangent (tanh). We implement the maxpooling layer instead of the average pooling as sub-sampling layer~\cite{krizhevsky_Imagenet_2017}. We observed that the performance is better when ReLU was configured with the maxpooling layer, resulting in higher classification accuracy after 50 epochs. We used the $1 \times 1$ convolutional filters (as suggested in~\cite{gupta_Convolutional_2017}) for the Adam~\cite{kingma_Variational_2015} optimizer. All the other parameters such as number of layers, convolutional kernel size, training algorithm, and the number of neurons in the final dense layer were all experimentally determined for our application.

The inputs to our CNN are the $32 \times 32$ image patches of primary cilia, extracted from both within AQP2 expressed regions and elsewhere within the microscopy images. We do not zero-pad the image patches, as we already select patch sizes that are much larger than a typical cilia length and thereby avoiding the additional computational cost. Next, two consecutive convolutional layers are used in the network. The first convolutional layer consists of 32 kernels of size $5 \times 5 \times 1$. The second convolutional layer consists of 64 kernels of size $5 \times 5 \times 32$. The sub-sampling layer is set as the maximum values in non-overlapping windows of size $2 \times 2$ (stride of 2). This reduces the size of the output of each convolutional layer by half. After the two convolutional and sub-sampling layers, we use a ReLU, where the activation $y$ for a given input $x$ is obtained as
\begin{equation}
y = f(x) = \text{max}(0,x)
\label{Cilia.Eq.1}
\end{equation}
A graphical representation of the architecture of the proposed CNN model is shown in Fig~\ref{Cilia.Fig.2} (see the left side).
% $y = f(x) = \text{max}(0,x)$.

\subsubsection{\textbf{Random Decision Forests}}

A random forest classifier consists of a collection of decision tree classifiers combined together to predict the class label, where each tree is grown in some randomized fashion. Each decision tree classifier consists of decision (or split) nodes and prediction (or leaf) nodes. The prediction node of each tree in the random forest classifier are labeled by the posterior distribution over the image classes~\cite{bosch_Image_2007}. Each decision node contains a test that splits best the space of data to be classified. An image is classified by sending it down the decision tree and aggregating the reached leaf posterior distributions. Randomness is usually injected at two points during training: in sub-sampling the training data and in selecting node tests. Each tree within the random forest classifier is binary and grown in a top-down manner. We choose the binary test at each node by maximizing the information gain,
\begin{equation}
\Delta E = -\sum_{i}\frac{\mid Q_{i}\mid}{\mid Q\mid}E(Q_{i})
\label{Cilia.Eq.2}
\end{equation}
obtained by partitioning the training set $Q$ of image patches into two sets $Q_{i}$ according to a given test. Here $E(q)$ is the entropy of the set $q$ and $\mid \cdot\mid$ is the size of the set. We repeat this selection process for each decision node until it reaches a certain depth.

Suppose $T$ is the set of all trees, $C$ is the set of all classes, and $L$ is the set of all leaves for a give tree. During training the posterior probabilities $\left(P_{t,l}(Y(I) = c)\right)$ for each class $c \in C$ at each leaf node $l \in L$, are found for each tree $t \in T$. These probabilities are calculated as the ratio of the number of images $I$ of class $c$ that reach a leaf node $l$ to the total number of images that reach that leaf node $l$. $Y(I)$ is the class label $c$ for image $I$. During test time, we pass a new image through every decision tree until it reaches a prediction (or leaf) node, average all the posterior probabilities and classify the image as
\begin{equation}
\hat{Y}(I) = \underset{c}{\arg \max}\left\{\frac{1}{\mid T\mid}\sum_{t=1}^{\mid T\mid} P_{t,l}(Y(I) = c)\right\}
\label{Cilia.Eq.3}
\end{equation}
where $l$ is the leaf node reached by the image $I$ in tree $t$. A graphical representation of the proposed random forest classifier is shown in Fig.~\ref{Cilia.Fig.2} (see the right side).

\section{Experiments and Results}

The Leica TCS SP5 II laser scanning confocal microscope (Leica Microsystems, Buffalo Grove, IL, USA) was used in this work to capture the images. A Plan-Neofluar lens with a magnification of 4$\textsf{x}$, numerical aperture = 0.9, and a pixel size of 0.4 $\mu$m in the x- and y-directions with automatic focusing was used to acquire the images. The size of each image in our dataset is $2048 \times 2048$. We used a total of 8 images consisting of a total of 2357 primary cilia, with 406 primary cilia within AQP2 expressed regions in the images and 1951 primary cilia elsewhere within the image. A careful manual detection and classification of all the 2357 primary cilia was performed by an expert and considered as ground truth for subsequent analysis. We compared our proposed classifier with an unsupervised \emph{k}-means classifier~\cite{dundar_Simplicity_2015} and a supervised multilayer perceptron (MLP) classifier~\cite{haykin_Neural_2009}.

\subsection{Algorithm Parameters}

The convolutional neural random forest classifier is implemented using TensorFlow~\cite{abadi_tensorflowlarge_2016}, and runs on an NVIDIA GeForce GTX TITAN X GPU with 8GB of memory. We used 70\% of the data for training and 30\% of the data for testing. A total of 284 primary cilia from AQP2 expressed regions and 1365 cilia from elsewhere within the images were used for training the algorithm. The classifier was trained using the stochastic gradient descent (SCD) algorithm, Adam~\cite{kingma_Adam_2014}, to efficiently optimize the weights of the CNN\@. The weights were normalized using initialization as proposed in~\cite{gupta_Convolutional_2017} and updated in a mini-batch scheme of 128 candidates. The biases were initialized with zero, and the learning rate was set to $\alpha = 0.001$. The exponential decay rates for the first and second moment estimates were set as $\beta_{1} = 0.9$ and $\beta_{2} = 0.999$, respectively. We used an $\epsilon = 10^{-8}$ to prevent division by zero. A dropout rate of 0.2 was implemented as regularization, applied to the output of the last convolutional layer and the dense layer to avoid overfitting. Finally, we used an epoch size of 50. The softmax loss (cross-entropy error loss) was utilized to measure the error loss. We used 100 estimators and a keep rate of $\gamma = 10^{-4}$ for the random forests algorithm. A 5-fold cross-validation was used during training.%

\subsection{Performance Evaluation}

The \emph{k}-means, MLP, and the proposed convolutional neural random forest classifier were tested on 30\% of the whole data consisting of 122 primary cilia in AQP2 expressed regions and 586 cilia elsewhere within the images. We evaluated all the algorithms using the conventional metrics that have been used for evaluation of classification algorithms, namely precision $P$, recall $R$, receiver operating characteristic (ROC) curves, area under the curve (AUC), and coverage measure ($F_{\beta}$-score).

Precision $P$ and recall $R$ are given by
\begin{equation}
P = \frac{\text{TP}}{\text{TP} + \text{FP}}, \quad R = \frac{\text{TP}}{\text{TP} + \text{FN}}
\label{Cilia.Eq.4}
\end{equation}
where TP is the number of true positive classifications, FP is the number of false positive classifications, and FN is the number of false negative classifications.

An ROC curve is a plot between the true positive rate (a.k.a.\ sensitivity or recall ($R$)), which is defined by~(\ref{Cilia.Eq.4}), and the false positive rate (a.k.a.\ complement of specificity), which is defined as $\text{FP}/(\text{FP} + \text{FN})$.

The coverage measure, also commonly known as the $F_{\beta}$-score is defined by
\begin{equation}
F_{\beta} = \left(1 + \beta^{2}\right)\frac{PR}{\left(\beta^{2}P\right) + R}
\label{Cilia.Eq.5}
\end{equation}
We use $F_{1}$ (i.e., $\beta = 1$) as this is the most common choice for this type of evaluation.

The AUC is the average of precision $P(R)$ over the interval ($0 \leq R \leq 1$), where $P(R)$ is a function of recall $R$. It is given by
\begin{equation}
\text{AUC} = \int_{0}^{1} P(R)dR.
\label{Cilia.Eq.6}
\end{equation}
The best classification algorithm among several alternatives is commonly defined as the one that maximizes either the AUC  or the $F_{\beta}$-score.
%----------------------------------------------- TABLE 1------------------------------------------------------%
\begin{table}[!t]
\caption{Performance of the Classification Algorithms}
\label{Cilia.Table.1}
  \begin{center}
	\renewcommand{\arraystretch}{1.7}
	\begin{tabular}{>{\centering} m{1.4cm} >{\centering} m{1.7cm}  >{\centering}m{1.35cm} >{\centering} m{0.9cm} >{\centering}m{1.25cm}}
	\hline
	\rowcolor[gray] {0.8}\textbf{Methods} & \textbf{Precision ($P$)} & \textbf{Recall ($R$)} & \textbf{AUC} & \textbf{$F_{\beta}$-score} \tabularnewline \hline
	Our Method &  0.9143 & 0.9062 & 0.8514 & 0.9102 \tabularnewline
	MLP & 0.8234 & 0.8239 & 0.8102 & 0.8237 \tabularnewline
	\emph{k}-means & 0.7961 & 0.8112 & 0.7891 & 0.8035 \tabularnewline
	\hline
	\end{tabular}
   \end{center}
\vspace{-4mm}
\end{table}
%-------------------------------------------------------------------------------------------------------------------%

Table\,\ref{Cilia.Table.1} shows the average precision ($P$), recall ($R$), AUC, and $F_{\beta}$-score values for all the classification algorithms on the test data. From Table\,\ref{Cilia.Table.1} we observe that the $F_{\beta}$-score of the proposed method is 10.67 percentage points greater than the \emph{k}-means classifier, and is 8.65 percentage points greater than the MLP classifier. Table\,\ref{Cilia.Table.1} also shows that the proposed classifier has the largest AUC among all the evaluated methods. Fig.~\ref{Cilia.Fig.3} shows the ROC curves for all the methods under comparison. From Fig.~\ref{Cilia.Fig.3}, we observe that the proposed method has better classification accuracy compared to the other automated methods at all points along the curve.

\section{Conclusion}

Accurate detection and classification of primary cilia in microscopy images is a challenging task. We propose a convolutional neural random forest classifier to classify primary cilia to determine whether they lie within an AQP2 expressed region or elsewhere within the microscopy images. We have shown how to model and train random forests, usable as alternative classifiers for batch learning in (deep) convolutional neural networks. Our approach combines the representation learning power of CNNs along with the divide-and-conquer principle of decision trees. We applied the proposed classifier to the problem of primary cilia classification in microscopy images and compared it with two methods, an unsupervised k-means classifier and a supervised MLP classifier. The results show that the proposed algorithm achieves better classification accuracy compared to the other two classifiers in terms of various figures of merit such as AUC, and $F_{\beta}$-score.
%
%For future we plan to incorporate an end-to-end learning framework for the convolutional neural random forests by introducing stochastic differentiable decision trees, enabling the split node parameters to be learned via back-propagation. Doing so will lead to further improvements in the classifier.
%
%\section*{Acknowledgment}
%
%The authors would like to thank Jonathan T. Gill for the numerous fruthfull discussions and insights about using TensorFlow. We would also like to thank Jianbo Shao for helping in manual segmentation of the primary cilia images used for ground truth evaluation.
%----------------------------------------------- FIGURE 3------------------------------------------------------%
\begin{figure}[!htbp]
	\centering
	\includegraphics[width= 3.5in, height=2.5in]{\figurepath{fig3}}
	\caption{ROC curves for all the classification methods.}
	\label{Cilia.Fig.3}
	\vspace{-4mm}
\end{figure}
%-----------------------------------------------------------------------------------------------------------------%


