\chapter{Classification of Primary Cilia in Microscopy Images Using Convolutional Neural Random Forests}

\begin{abstract}
Accurate detection and classification of primary cilia in microscopy images is an essential and fundamental task for many biological studies including diagnosis of primary ciliary dyskinesia. Manual detection and classification of individual primary cilia by visual inspection is time consuming, and prone to induce subjective bias. However, automation of this process is challenging as well, due to clutter, bleed-through, imaging noise, and the similar characteristics of the non-cilia candidates present within the image. We propose a convolutional neural random forest classifier that combines a convolutional neural network with random decision forests to classify the primary cilia in fluorescence microscopy images. We compare the performance of the proposed classifier with that of an unsupervised \emph{k}-means classifier and a supervised multi-layer perceptron classifier on real data consisting of 8 representative cilia images, containing more than 2300 primary cilia using precision/recall rates, ROC curves, AUC, and F$_{\beta}$-score for classification accuracy. Results show that our proposed classifier achieves better classification accuracy.

\textbf{KEYWORDS:\ } Image classification, convolutional neural network, random forests, primary cilia, confocal microscopy.
\end{abstract}

\section{Introduction}
Primary cilia are curvilinear non-motile sensory organelles protruding from the surface of many eukaryotic cells that are involved in many cell development and physiological processes. Recent research~\cite{miyoshi_Lithium_2009} has shown that primary cilia length in mammalian cells may change due to extracellular environment stimuli, such as renal injury~\cite{verghese_Renal_2009} or external signaling modules. It has also been reported that renal primary cilia are involved in modulation of the mechanistic target of rapamycin (mTOR) pathway. Furthermore, it has been demonstrated that lithium treatment activates the mTOR pathway in renal collecting duct cells expressing aquaporin 2 (AQP2)~\cite{gao_Rapamycin_2013}. The collecting duct cells in the kidney express primary cilia, as shown in Fig.~\ref{Cilia.Fig.1} \(a\). There is a significant interest developing an automated classifier to accurately and rapidly distinguish between primary cilia located in the AQP2 expressed region and primary cilia located elsewhere.

Manual detection and classification of primary cilia within tissues usually involves a large amount of time, especially for large-scale data processing, and is prone to multiple errors caused by background clutter, non-uniform illumination, imaging noise, and subjective bias. This serves as a motivation to develop an automatic algorithm capable of classifying the primary cilia of interest within the fluorescently labeled microscopy images. In order to classify the primary cilia, we need to first find all the cilia locations within the microscopy images. There are a number of curvilinear-structure detection methods such as the top-hat transform-based detector, ridge detector, steerable detector and the multi-scale variance stabilizing transform (MS-VST) detector that are capable of detecting all the cilia locations within the microscopy image~\cite{ram_dissertation_2017, ram_Vehicle_2016, ram_Three_2018}. Nevertheless, in this paper we analyze the already detected primary cilia, and we focus on classifying them as belonging to an AQP2 expressed region or elsewhere in the microscopy images.
