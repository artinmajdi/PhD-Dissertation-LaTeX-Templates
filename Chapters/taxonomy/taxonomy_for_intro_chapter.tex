\chapter{A Hierarchical Multilabel Classification Method for Enhanced Thoracic Disease Diagnosis in Chest Radiography}

\begin{abstract}
Accurate diagnosis of thoracic diseases from chest radiographs is a challenging task that can lead to diagnostic errors and negative patient outcomes. In this study, we propose a novel hierarchical multilabel classification technique that utilizes the taxonomical relationship between different pathologies to improve classification accuracy. Two methods are proposed to encompass both scenarios where the ground truth is available (referred to as ``loss'' in this paper) and when it is not (referred to as ``logit'').  The proposed methods leverage a predefined disease taxonomy to account for interrelationships among diseases, thereby augmenting their generalizability to novel tasks.  The ``logit'' approach can be seamlessly integrated into existing pre-trained models without the need for re-optimization, ensuring efficiency and broad applicability. The ``loss'' approach can be incorporated into the existing technique during the training phase by modifying the loss function.  To evaluate the effectiveness of the proposed technique, experiments were conducted on three diverse and publicly available chest radiograph datasets (CheXpert, PadChest, and NIH Chest-Xray14). The results demonstrate that the proposed technique significantly improves the accuracy and interpretability of machine learning models for thoracic disease on chest radiography. This approach has the potential to promote an accurate and efficient diagnosis by providing radiologists with an additional layer of decision support, ultimately leading to better patient outcomes.

\textbf{KEYWORDS:\ } Chest radiography, hierarchical classification, disease taxonomy, multilabel classification, conditional loss function, diagnostic errors, machine learning, medical imaging
\end{abstract}


\section{Introduction}

The timely diagnosis and effective treatment of diseases depend on the precise and efficient detection of anomalies in medical imaging. Deep learning techniques have made substantial progress in the medical imaging domain, exhibiting impressive success across various applications~\cite{litjens_Survey_2017a}. Nonetheless, conventional classification methods primarily designed for single-label predictions struggle with multi-label classification, which requires predicting multiple labels for each input sample.

Chest radiography (CXR) is a prevalent radiological examination for diagnosing lung and heart disorders, constituting a significant share of ordered imaging studies. Swift and accurate detection of different conditions, such as pneumothorax, is crucial for optimal patient care~\cite{bellaviti_Increased_2016}. However, interpreting CXRs can be challenging due to similarities between multiple diseases, which may result in misinterpretations even by experienced radiologists~\cite{delrue_Difficulties_2011}. Consequently, devising an accurate system to identify and localize common thoracic diseases can aid radiologists in minimizing diagnostic errors~\cite{crisp_Global_2014,silverstein_Most_2016}.

Convolutional neural networks (CNNs) exhibit potential for learning intricate relationships between image objects. However, their training necessitates vast amounts of labeled data, which can be both expensive and time-consuming to acquire. Despite these challenges, deep learning techniques have become increasingly popular in medical imaging, especially in radiology, due to their ability to perform complex tasks with minimal human intervention~\cite{jaderberg_Spatial_2015}. Progress in natural language processing (NLP) has enabled the collection of extensive annotated datasets such as ChestX-ray8~\cite{wang_ChestXRay8_2017}, MIMIC-CXR~\cite{johnson_MIMICCXR_2019}, and CheXpert (Irvin et al., 2019b), allowing researchers to develop more efficient and robust supervised learning algorithms.

Regarding multi-label classification, common methods like the One-vs-All (OVA) approach exhibit limitations, including high computational complexity and an inability to capture intricate label relationships~\cite{tsoumakas_MultiLabel_2007}. Although recent advances in deep learning have facilitated the creation of CAD systems capable of classifying and localizing prevalent thoracic diseases using CXR images, most of these techniques have concentrated on specific diseases~\cite{jaiswal_Identifying_2019,lakhani_Deep_2017,pasa_Efficient_2019,ausawalaithong_Automatic_2018}, leaving ample opportunities to investigate a unified deep learning framework that can efficiently detect a broad spectrum of common thoracic diseases.

This paper aims to tackle the challenges of multi-label classification within the realm of medical imaging by introducing a hierarchical framework that can be employed in a transfer learning approach without necessitating costly computational resources. The rest of this paper is structured as follows. Section 2 discusses related work on multi-label classification and hierarchical loss functions; Section 3 describes our proposed method for integrating label hierarchy into multi-label loss functions; Section 4 presents experimental results using the chest radiograph dataset; and Section 5 concludes the paper and outlines future research directions.
